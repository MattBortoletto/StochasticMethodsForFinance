\lesson{34}{28/05/2020}Now we have all the ingredients to find the characteristic function (the Laplace transform) of the asset price.
\begin{proposition}
    The characteristic function of the asset price is given by
    \begin{equation}\label{characteristic-func}
        \psi_{\gamma,t}(\tau) = \exp{\Tr(A(\tau)\Sigma_t)+b(\tau)Y(t)+c(\tau)}
    \end{equation}
    where
    \begin{itemize}
        \item $A(\tau) = A_{22}(\tau)^{-1}A_{21}(\tau)$ is the solution of the Riccati matrix equation with
        \begin{equation}
            \mqty(A_{11}(\tau) & A_{12}(\tau) \\ A_{21}(\tau) & A_{22}(\tau)) = \mqty(M & -2Q^TQ \\ \tfrac{\gamma(\gamma -1)}{2}\mathds{1}_n & -(M^T+2\gamma RQ));
        \end{equation}
        \item $b(\tau) = \gamma$;
        \item $c(\tau) = \tfrac{\beta}{2}\Tr(\ln F(\tau)+(M^T+2\gamma RQ)\tau)+\gamma r\tau$, where $F$ comes from the linearization of the Riccati equation: $A=F^{-1}G$.
    \end{itemize}
\end{proposition}
\begin{proof}
    If we consider the Feynman-Kac PDE we get:
    \begin{align*}
        \pdv{\psi}{\tau} &= \left(r-\frac{1}{2}-\Tr(\Sigma)\right)\pdv{\psi}{y} + \frac{1}{2}\Tr(\Sigma)\pdv[2]{\psi}{y} + \\
        &\qquad
        + \Tr[(\beta Q^TQ + M\Sigma + \Sigma M^T)D\psi + 2\Sigma DQ^TQD\psi] + 2\Tr(\Sigma RQD)\pdv{\psi}{y}
    \end{align*}
    Substituting the candidate solution \eqref{characteristic-func} we get
    \begin{align*}
        0 &= -\Tr(\dot{A}(\tau)\Sigma) - \dot{b}(\tau)Y - \dot{c}(\tau) + \Tr\left[(\beta Q^TQ + M\Sigma + \Sigma M^T)A(\tau)\right. + \\
        &\qquad
        + \left. 2\Sigma A(\tau)Q^TQA(\tau)+2\Sigma RQA(\tau)b(\tau)\right] + \left(r-\frac{1}{2}-\Tr(\Sigma)\right)b(\tau) + \frac{1}{2}\Tr(\Sigma)b^2(\tau)
    \end{align*}
    with the boundary conditions
    \begin{align*}
        A(0) &= 0 \in M_n \\
        b(0) &= \gamma \in \mathbb{R} \\
        c(0) &= 0 \in \mathbb{R}
    \end{align*}
    By identifying the coefficients we get
    \begin{align*}
        \dot{b}(\tau) = 0 \qquad\Rightarrow\qquad b(\tau) \equiv \gamma \quad\forall t
    \end{align*}
    \begin{align*}
        \begin{cases}
            \dot{A}(\tau) = AM + (M^T+2\gamma RQ)A + 2AQ^TQA + \frac{\gamma(\gamma-1)}{2}\mathds{1}_n \\
            A(0) = 0
        \end{cases}
    \end{align*}
    \begin{equation*}
        \begin{cases}
            \dot{c}(\tau) = \Tr(\beta Q^TQA(\tau)) + \gamma r \\
            c(0) = 0
        \end{cases}
    \end{equation*}
    Then we linearize by taking $A = F^{-1}G$ and we solve the Riccati equation for $A$ and plug the result into the equation for $c$. Recalling that
    \begin{equation*}
        G = -\frac{1}{2}\left(\dot{F}+\dot{F}(M^T+2\gamma RQ)\right)(Q^TQ)^{-1}
    \end{equation*}
    we get
    \begin{equation*}
        \dot{c}(\tau) = -\frac{\beta}{2}\Tr(F^{-1}\dot{F} + M^T + 2\gamma RQ) + \gamma r
    \end{equation*}
    and integrating we get eq. \eqref{characteristic-func}.
\end{proof}
We now have a model for the underlying with the volatility driven by a matrix. What is the advantage with respect to a multi-factor Hestom model?

\subsection{Comparison between Wishard and \texorpdfstring{$A_2(3)$}{A_2(3)}-Heston model}
The $A_2(3)$-Heston model is a Heston model with three factors, with two of them positive. For example, consider the 2-Heston specification with two factors
\begin{align*}
    \dd X^1 = K^1(\theta^1-X^1)\dd t + \epsilon^1\sqrt{X^1}\,\dd W^1 \\
    \dd X^2 = K^2(\theta^2 - X^2)\dd t + \epsilon^2\sqrt{X^2}\,\dd W^2
\end{align*}
and introduce the log-asset price
\begin{align*}
    \dd Y(t) &= \left(r-\frac{1}{2}(X^1+X^2)\right)\dd t + \rho_1\sqrt{X^1}\,\dd W^1 + \rho_2\sqrt{X^2}\,\dd W^2 + \\
    &\qquad
    +\sqrt{(1-\rho_1^2)X^1 + (1-\rho_2^2)X^2}\,\dd B
\end{align*}

17:34

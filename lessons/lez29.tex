\lesson{29}{15/05/2020} As usual, the underlying satisfies the SDE
\begin{equation*}
    \frac{\dd S}{S} = r\,dd t + \sqrt{v(t)}\,\dd W(t)
\end{equation*}
and the volatility follows the CIR process dynamics
\begin{equation}
    \dd v = K(\theta-v(t))\,\dd t + \eta\sqrt{v(t)}(\rho\,\dd W(t) + \sqrt{1-\rho^2}\,\dd W(t)^{\perp})
\end{equation}
which is an affine process. Let's shift to the dynamics of the log-price (the returns) $x(t)=\ln S(t)$. Using Itô, we get
\begin{equation}
    \dd x(t) = \left(r-\frac{1}{2}v(t)\right)\,\dd t + \sqrt{v(t)}\,\dd W(t)
\end{equation}
Since $v$ is embedded in the dynamics of $x$, we have to consider it when we compute the expected value. Moreover, since both $x$ and $v$ are affine, we guess that the expected value has a joint affine structure:
\begin{equation}
    \expect_{t,x,v}[e^{izx_T}] = e^{A(t,T)+B(t,T)v(t)+C(t,T)x(t)} \equiv G(\tau,z,x,v), \qquad \tau = T-t
\end{equation}
In order to transform this conditional expected value into a PDE we use the Feynman-Kac formula:
\begin{align}\label{FKh}
    \notag 0 &= - \pdv{G}{\tau} + \left(r-\frac{1}{2}v\right)\pdv{G}{x} + K(\theta-v)\pdv{G}{v} + \frac{1}{2}\underbrace{v}_{\expval{x}}\pdv[2]{G}{x} + \\
    &\quad + \frac{1}{2}\underbrace{\eta^2 v}_{\expval{v}}\pdv[2]{G}{v} + \underbrace{\eta\rho v}_{\expval{x,v}} \pdv{G}{x}{v}
\end{align}
with initial condition
\begin{equation*}
    G(0,z,x,v) = e^{izx}
\end{equation*}
Since the couple $(x,v)$ is affine, we can guess that the solution can be written as a function of the time difference $\tau=T-t$:
\begin{equation*}
    G(\tau, z,x,v) = e^{A(\tau)+B(\tau)v(t)+C(\tau)x(t)}
\end{equation*}
so we save one parameter. This is our candidate solution, so now we have to impose the structure we want:
\begin{align*}
    -\pdv{G}{\tau} = -G(\dot{A}+\dot{B}v+\dot{C}x)
\end{align*}
\begin{align*}
    \pdv{G}{x} = CG, \qquad \pdv[2]{G}{x} = C^2G, \qquad \pdv{G}{v} = BG
\end{align*}
\begin{align*}
    \pdv[2]{G}{v} = B^2G, \qquad \pdv{G}{x}{v} = BCG.
\end{align*}
Substituting in \eqref{FKh} we get:
\begin{align}
    \notag 0 &= -\cancel{G}(\dot{A}+\dot{B}v+\dot{C}x) + \left(r-\frac{1}{2}v\right)C\cancel{G} + K(\theta-v)B\cancel{G} + \frac{1}{2}vC^2\cancel{G} + \\
    &\quad + \frac{1}{2}\eta^2vB^2\cancel{G} + \eta\rho v BC\cancel{G} \qquad \forall(x,v)
\end{align}
Separately identifying the coefficients of $x$ and $v$, we get the following ODEs. Coefficients of $x$:
\begin{equation*}
    \begin{cases}
    -\dot{C} = 0 \\
    C(0) = iz
    \end{cases} \qquad \Rightarrow \qquad C(\tau) = iz
\end{equation*}
Coefficients of $v$:
\begin{equation*}
    \begin{cases}
    -\dot{B} -\frac{1}{2}iz - KB + \frac{1}{2}(iz^2) + \frac{1}{2}\eta^2B^2 + \eta\rho B(iz) = 0 \\
    B(0) = 0
    \end{cases}
\end{equation*}
This is a true Riccati ODE with constant coefficients. Then, if we consider the constant terms we get
\begin{equation*}
    \begin{cases}
    -\dot{A} + rC + K\theta B = -\dot{A} + riz + K\theta B = 0 \\
    A(0) = 0
    \end{cases}
\end{equation*}
Now we have to solve the Riccati equation. In order to do that we can double the dimension and linearize or write solution in terms of a derivative of a function divided by the function. This last approach comes from a property of the Riccati equation, that lives in a space were variables can be parametrized by ratios. Let's introduce the change of variable
\begin{equation}
    B(\tau) = - \frac{\dot{E}(\tau)}{\tfrac{\eta^2}{2}E(\tau)}
\end{equation}
which transforms the quadratic first order ODE into a linear second order ODE:
\begin{equation*}
    \dot{E} + (K-\rho\eta iz)\dot{E} - \frac{\eta^2}{4}(iz+z^2)E = 0
\end{equation*}
and let's look for soluions of the form $e^{\lambda \tau}$. Substituting we get
\begin{equation*}
    \lambda^2 + (K-\rho\eta iz) \lambda - \frac{\eta^2}{4}(iz-z^2) = 0
\end{equation*}
Solving this quadratic equation we get
\begin{equation*}
    \lambda_{1,2} = \frac{-(K-\rho\eta iz)^2 \pm \sqrt{\Delta}}{2}
\end{equation*}
where
\begin{equation*}
    \Delta = (K-\rho\eta iz)^2 + \eta^2(iz+z^2).
\end{equation*}
Now, if we introduce the two auxiliary functions
\begin{equation*}
    \psi^{\pm} = -(K-\rho\eta iz) \pm \sqrt{\Delta}
\end{equation*}
such that
\begin{equation*}
    \psi^+-\psi^- = 2\sqrt{\Delta}, \qquad \psi^+-\psi^- = -\eta^2(iz+z^2),
\end{equation*}
the general solution can be written as
\begin{equation}
    E(\tau) = \alpha_1 e^{\frac{\psi^+}{2}\tau} + \alpha_2 e^{\frac{\psi^-}{2}\tau}
\end{equation}
where $\alpha_1, \alpha_2$ are two constants to determine according to the initial conditions
\begin{equation*}
    E(0) = \alpha_1 + \alpha_2
\end{equation*}
\begin{equation*}
    \dot{E}(t) = \frac{1}{2}\psi^+\alpha_1 e^{\frac{\psi^+}{2}\tau} + \frac{1}{2}\psi^-\alpha_2 e^{\frac{\psi^-}{2}\tau} \qquad \Rightarrow \qquad \dot{E}(0) = \frac{1}{2}\psi^+\alpha_1 + \frac{1}{2}\psi^-\alpha_2
\end{equation*}
This leads to
\begin{equation*}
    \alpha_1 = \frac{\psi^- E(0)}{2\sqrt{\Delta}}, \qquad \alpha_2 = \frac{\psi^+ E(0)}{2\sqrt{\Delta}}
\end{equation*}
Finally, we can get an expression for $B$:
\begin{align}
    \notag B(\tau) &= -\frac{
    \psi^+ \psi^- \left(
    e^{\frac{1}{2}\psi^+\tau} - e^{\frac{1}{2}\psi^-\tau}
    \right)
    }{
    \eta^2\left(
    \psi^- e^{\frac{1}{2}\psi^+\tau} - \psi^+ e^{\frac{1}{2}\psi^-\tau}
    \right)} \\
    &=
    -(iz+z^2)\frac{e^{\frac{1}{2}\psi^+\tau} - e^{\frac{1}{2}\psi^-\tau}}{\psi^+ e^{\frac{1}{2}\psi^-\tau} - \psi^- e^{\frac{1}{2}\psi^+\tau}}
\end{align}
Now we can find $A$ by integrating:
\begin{align}
    \notag A(\tau) &= \int_0^{\tau} k\theta B(s)\,\dd s + riz\tau \\
    &=
    \notag -\frac{2K\theta}{\eta^2} \int_0^{\tau} \frac{\dot{E}(s)}{E(s)}\,\dd s + riz\tau \\
    &=
    \notag -\frac{2K\theta}{\eta^2}\ln\left(\frac{E(\tau)}{E(0)}\right) + riz\tau \\
    &=
    -\frac{k\theta}{\eta^2}\left[2\ln\left(\frac{-\psi^- + \psi^+ e^{-\sqrt{\Delta}}}{2\sqrt{\Delta}}\right) + \psi^+\tau \right]
\end{align} %fine parte 1
Notice that this expression in completely explicit.

\section{The Carr-Madan approach}
Consider the specific case of a call option. The main idea of this approach is that a call option is a function not only of the spot but also of the strike. So, instead of considering the Laplace FT (LFT) and the anti-transform of the density function of the underlying (which is unknown) we can take the LFT of the price of the call and then take the inverse transform. The LFT is performed on the strike because we want the price of the call for many strikes (in view of the calibration of the model). So, if we parametrize the price of a call as $\call(x,l,T)$ where $x = \ln S$ and $l = \ln K$, then we have that
\begin{equation}
    \call(x,l,T) = e^{-r(T-t)}\int_{l}^{+\infty} (e^y - e^l)f_x(y)\,\dd y
\end{equation}
where $f_x(y)$ is the underlying density function. Notice that the call, as a function of $l$, is not $L^1$, in fact
\begin{equation*}
    \lim_{l\to-\infty}\call = e^{-r(T-t)}\expect_t[e^{x_T}]
\end{equation*}
because for $l\to-\infty$ we have $K\to0$ and the price of the call coincides with the underlying. This is the reason why in order to get the integrability of the call price we can't take the LFT but we need something more. So, we take the LFT with respect to $l$. If we consider a generic complex number $z = \xi + i\eta$, the LFT is given by
\begin{equation*}
    \widehat{\call}(z) = \widehat{\call}(\xi+i\eta) = \int_\mathbb{R} e^{i(\xi+i\eta)l}\call(x,l,t)\,\dd l
\end{equation*}
Putting the discount factor on the other side we have that
\begin{align*}
    e^{r(T-t)}\widehat{\call}(z) &= \int_{\mathbb{R}} e^{i(\xi+i\eta)l} \left( \int_{l}^{+\infty} (e^y - e^l)f_x(y)\,\dd y \right)\,\dd l \\
    \intertext{Let's use Fubini:}
    &=
    \int_{\mathbb{R}}\dd y \int_{-\infty}^y e^{i(\xi+i\eta)l} (e^y - e^l)f_x(y)\,\dd l \\
    &=
    \int_{\mathbb{R}}f_x(y)\,\dd y \int_{-\infty}^y e^{i(\xi+i\eta)l} (e^y - e^l)\,\dd l \\
    &=
    \int_{\mathbb{R}}f_x(y)\,\dd y \left(e^y \int_{-\infty}^y e^{i(\xi+i\eta)l}\,\dd l - \int_{-\infty}^y e^{i(\xi+i\eta)l + l}\right)\,\dd l \\
    \intertext{Since the real part of the integrand involves $il\cdot i\eta = -l\eta$, the integral converges only if $\eta<0$.}
    &=
    \frac{1}{i} \int_{\mathbb{R}}f_x(y) e^{y + iy(\xi+i\eta)}\left(\frac{1}{\xi+i\eta} - \frac{1}{\xi+i(\eta-1)}\right)\,\dd y \\
    &=
    \frac{1}{i} \hat{f}_x(\xi+i(\eta-1))\left(\frac{1}{\xi+i\eta} - \frac{1}{\xi+i(\eta-1)}\right) \\
    &=
    \hat{f}_x(\xi+i(\eta-1))\frac{1}{\eta^2-\eta-\xi^2+i\xi(1-2\eta)}
\end{align*}
\begin{remark}
    In order to use Fubini we dominate the absolute value of the integrand for both $y$ and $l$. In other words, we have to bound the quantity
    \begin{equation*}
        \abs{e^{i(\xi+i\eta)l} (e^y - e^l)f_x(y)\mathds{1}_{y>l}}
    \end{equation*}
    Regarding the imaginary part there is no problem because the function is circular, so we can bound with 1 the exponential. Let's focus on the real part. If we replace the minus sign with the plus, we get the trivial inequality
    \begin{equation*}
        \abs{e^{-\eta l} (e^y - e^l)f_x(y)\mathds{1}_{y>l}} \le \mathds{1}_{y>l}f_x(y)e^ye^{-\eta l} + \mathds{1}_{y>l}f_x(y)e^{-l(\eta-1)}
    \end{equation*}
    Now, let's integrate with respect to $l$:
    \begin{equation*}
        e^y f_x(y) \int_{-\infty}^y e^{-\eta l}\,\dd l + f_x(y) \int_{-\infty}^y e^{-l(\eta-1)}\,\dd l
    \end{equation*}
    We see that if $\eta<0$ then the integrand is bounded, provided that
    \begin{equation*}
        \int_{\mathbb{R}}\left(\frac{e^{y-\eta y}}{\abs{\eta}}f_x(y)\right) < \infty
    \end{equation*}
    which is equivalent to
    \begin{equation*}
        \mathbb{E}\left[(e^y)^{1-\eta}\right] < \infty
    \end{equation*}
    Since $e^y = S(T)$, this condition asks that the underlying has a finite $(1-\eta)$-moment, with $1-\eta>1$ (since $\eta$ is negative). Once again, we face the problem of moment explosion.
\end{remark}
So, if we take $\alpha = -\eta >0$ such that $\mathbb{E}[S^{1+\alpha}(T)]<\infty$, then the LFT of the call price is given by
\begin{align}
    \notag\widehat{\call}(z) &= \widehat{\call}(\xi+i\eta) = \widehat{\call}(\xi-i\alpha) \\
    &=
    e^{-r(T-t)}f_x(\xi - i(\alpha+1))\frac{1}{\alpha^2+\alpha-\xi^2+i(2\alpha-1)}
\end{align}
In order to get the call price, we have to take the inverse LFT. Exploiting the symmetry of the integrand we can consider only the integral from $0$ to $\infty$:
\begin{align}
    \call(x,l,T) = \frac{e^{-\alpha l}}{\pi}\int_0^{\infty} \Re{e^{-i\xi l}\widehat{\call}_x(\xi-i\alpha)}\,\dd\xi
\end{align}
In summary:
\begin{itemize}
    \item Fix $l=\ln K$
    \item Fix $\alpha >0$ so that there is no moment explosion
    \item The price of the call at time $t=0$ is given by
    \begin{align}
        \notag\call_0 &= \frac{e^{-\alpha l -rT}}{\pi}\int_0^{\infty} \Re{e^{-itl}\frac{\hat{f}_x(t-i(\alpha+1))e^{i(t-i(\alpha+1))(\ln S(0)+rT)}}{(\alpha+it)(\alpha+1+it)}}\,\dd t \\
        &=
        \frac{e^{-\alpha l -rT}}{\pi}
        \int_0^{\infty}
        \Re{
        \frac{
        e^{-itl+i(t-i(\alpha+1))(\ln S(0)+rT)}}{(\alpha+it)(\alpha+1+it)}
        \hat{f}_x(t-i(\alpha+1))
        }\,\dd t
    \end{align}
\end{itemize}

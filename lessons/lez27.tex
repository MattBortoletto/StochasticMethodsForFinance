\section{Affine term structures}\lesson{27}{13/05/2020} %Bjork ch 24.3
When we considered the short interest rate we said that the drift and the diffusion were affine functions of the short interest rate itself. Now, we have to deduce this fact from the properties of the pricing function of the zero coupon bond. So, the starting point is the price of the zero coupon bond:
\begin{align*}
    p(t,T) &= \expect_t\left[e^{-\int_t^T r(s)\,\dd s}\right] \\
    &=
    F(t, r(t), T)
\end{align*}
We introduced the Feynman-Kac methodology in order to solve a PDE in the case in which we do not have a candidate solution. In fact, thanks to the Feynman-Kac methodology we can transform the problem of finding the solution of a deterministic PDE into the problem of computing an expected value. \\
Now the situation is a bit different. We can write the price of the zero coupon bond as an expected value but in general we cannot compute it because we don't know the probability distribution of $r(t)$. So, we have to find a candidate solution of the PDE.\\
Suppose that
\begin{equation}
    p(t,T) = e^{A(t,T)-B(t,T)r(t)}
\end{equation}
where $A(t,T)$ and $B(t,T)$ are two deterministic functions and the conventional minus sign is coherent with the fact that if there is a rise up of the interest rate the price of the zero coupon bond will decrease. Then, the model associated to the short rate $r(t)$ is said to possess an \emph{affine term structure} (ATS).
Then, the question is: if the model has an ATS, which is the dynamics of the process? Let's consider the case
\begin{equation*}
    r(t) = x
\end{equation*}
and assume a generic SDE for $r$:
\begin{equation}
    \dd r(t) = \mu(t,r(t))\,\dd t + \sigma(t,r(t))\,\dd W(t)
\end{equation}
where measurability and integrability are enough to find a strong solution of this equation. Imposing the candidate solution
\begin{equation*}
    p(t,T) = e^{A(t,T)-B(t,T)r(t)}
\end{equation*}
to the Feynman-Kac equation
\begin{align*}
    \begin{cases}
    \pdv{F}{t} + \mu \pdv{F}{r} + \frac{1}{2}\sigma^2\pdv[2]{F}{r} - rF = 0 \\
    F(T,r) = 1
    \end{cases}
\end{align*}
we get
\begin{align*}
    \pdv{F}{t} &= F\left(\pdv{A}{t} - \pdv{B}{t}r\right) \\
    \pdv[2]{F}{r} &= F\cdot B^2
\end{align*}
and so
\begin{align*}
    \begin{cases}
    F\left(\pdv{A}{t} - \pdv{B}{t}r\right) - B\mu F + \frac{1}{2}\sigma^2 B^2F -rF = 0 \\
    e^{A(T,T)-B(T,T)r(T)} = 1.
    \end{cases}
\end{align*}
From the second equation we have that
\begin{equation}
        A(T,T) = 0, \qquad B(T,T) = 0
\end{equation}
Simplifying the first equation we get
\begin{equation*}
    \left(\pdv{A}{t} - \pdv{B}{t}r\right) - B\mu + \frac{1}{2}\sigma^2 B^2 -r = 0
\end{equation*}
Notice that there is a linear dependence on $r$ and then a generic dependence on $\mu$ and $\sigma^2$. If $\mu$ and $\sigma^2$ are both affine (i.e. linear plus a constant) functions of $r$, with possibly time dependent coefficients, then the differential equation becomes a separable.
Assume thus that $\mu$ and $\sigma$ assume the following form
\begin{equation*}
    \mu(t,r) = \alpha(t)r + \beta(t), \qquad \sigma(t,r) = \sqrt{\gamma(t)r + \delta(t)}.
\end{equation*}
Substituting we end up with
\begin{equation*}
    \dot{A} - \beta B + \frac{1}{2}\delta B^2 - \left(1+\dot{B} + \alpha B - \frac{1}{2}\gamma B^2\right)r = 0
\end{equation*}
This equation holds for all $t, T$ and $r$, so let us consider it for a fixed choice of $T$ and $t$. Since the equation holds for all values of $r$ the coefficient of $r$ must be equal to zero. Thus we have the equation
\begin{equation}\label{Aeq}
    \begin{cases}
        \dot{B} + \alpha B - \frac{1}{2}\gamma B^2 = -1 \\
        B(T,T) = 0
    \end{cases}
\end{equation}
which is called \emph{Riccati quadratic ODE} and if $B$ is a symmetric squared matrix it has a closed form solution.
Since the $r$-term is zero we see that the other term must also vanish,
giving us the equation
\begin{equation}\label{Beq}
    \begin{cases}
    \dot{A} = \beta B - \frac{1}{2}\delta B^2 \\
    A(T,T) = 0
    \end{cases}
\end{equation}
Once we know the solution $B(t,T)$, solving the ODE for a is trivial.
\begin{example}{Vašíček model}{}{} % bjork 24.1.1
    In the Vašíček model
    \begin{equation*}
        \dd r = (b - ar)\dd t + \sigma\,\dd W
    \end{equation*}
    so we have $\beta = b$, $\alpha = -a$, $\gamma = 0$ and $\delta = \sigma^2$. Eqs. \eqref{Aeq} and \eqref{Beq} become
    \begin{equation}
    \begin{cases}\label{VBeq}
        \dot{B} - aB = -1 \\
        B(T,T) = 0
    \end{cases}
    \end{equation}
    \begin{equation}
    \begin{cases}\label{VAeq}
    \dot{A} = \beta B - \frac{1}{2}\delta B^2 \\
    A(T,T) = 0
    \end{cases}
    \end{equation}
    Equation \eqref{VBeq} is, for each fixed T, a simple linear ODE in the t-variable. It can solved as
    \begin{equation*}
        B(t,T) = \frac{1}{a}\left(1-e^{-a(T-t)}\right)
    \end{equation*}
    which is exactly what we obtained before by computing the expected value. Integrating \eqref{VAeq} we obtain
    \begin{equation*}
        A(t,T) = \frac{\sigma^2}{2}\int_t^T B^2(s,T)\,\dd s - b\int_t^T B(s,t)\,\dd s
    \end{equation*}
    and, substituting the expression for B above, we obtain
    \begin{equation*}
        A(t,T) = \frac{(B(t,T)-T+t)(ab-\tfrac{1}{2}\sigma^2)}{a^2} -            \frac{\sigma^2 B^2(t,T)}{4a}.
    \end{equation*}
    The bond prices are given by
    \begin{equation*}
        p(t,T) = e^{A(t,T)-B(t,T)r(t)}.
    \end{equation*}
\end{example}
\begin{example}{Ho-Lee model}{}{}
    For the Ho–Lee model
    \begin{equation*}
        \dd r(t) = \Theta(t)\,\dd t + \sigma\,\dd W(t)
    \end{equation*}
    so $alpha = \gamma = 0$. The ATS equations become
    \begin{equation*}
    \begin{cases}
        \dot{B} - aB = -1 \\
        B(T,T) = 0
    \end{cases}
    \begin{cases}
    \dot{A} = \beta B - \frac{1}{2}\delta B^2 \\
    A(T,T) = 0
    \end{cases}
    \end{equation*}
    which can be solved as
    \begin{align*}
        B(t,T) &= T-t \\
        A(t,T) &= \int_t^T \Theta(s)(s-T)\dd s + \frac{\sigma^2(T-t)^3}{6}
    \end{align*}
\end{example} % fine parte 1
\begin{example}{CIR model}{}{}
    In the CIR model
    \begin{equation*}
        \dd r(t) = a(b-r(t))\dd t + \sigma\sqrt{r(t)}\,\dd W(t)
    \end{equation*}
    Notice that even if $\sqrt{r(t)}$ is not Lipschitz, the existence of a strong solution is guaranteed by the fact that it is Hölder continuous, so the Yamada condition can be applied. Since $\gamma = 1$, we get a true Riccati equation:
    \begin{equation*}
    \begin{cases}
        \dot{B} - aB - \frac{1}{2}\sigma^2 B^2 = -1 \\
        B(T,T) = 0
    \end{cases}
    \begin{cases}
    \dot{A} = a b B\\
    A(T,T) = 0
    \end{cases}
    \end{equation*}
    where $\sigma = \sqrt{\gamma r + \delta}$. The solution of the Riccati equation is
    \begin{equation*}
        B(t,T) = \frac{2(e^{\epsilon(T-t)}-1)}{(\epsilon-a)(e^{\epsilon(T-t)}-1)+2\epsilon}
    \end{equation*}
    with
    \begin{equation*}
        \epsilon = \sqrt{a^2 + 2\sigma^2}
    \end{equation*}
\end{example}

\section{Stochastic volatility}
We already know that the B\&S model is not exact, in the sense that it is not able to reproduce the volatility smile. In order to go beyond, the idea is to assume the volatility to be random, driven by a stochastic process which in principle is (partically) correlated with the asset. Keeping the interest rate constant (we are not interest in the its fluctuations in the short term) and considering a stochastic volatility we can write the B\&S equation as
\begin{equation}
    \frac{\dd S}{S} = r\,\dd t + \sigma(t)\,\dd W^{\Qmeas}(t)
\end{equation}
It is possible to use the square root of the instantaneous variance instead of the volatility:
\begin{equation}
    \frac{\dd S}{S} = r\,\dd t + \sqrt{V(t)}\,\dd W^{\Qmeas}(t)
\end{equation}
One of the pioneering models for stochastic volatility is due to Heston (1993), who borrowed the CIR process for the interest rate and applied to the volatility:
\begin{equation}
    \dd V(t) = k(V(\infty)-V(t))\dd t + \eta\sqrt{V(t)}(\rho\,\dd W^{\Qmeas}(t) + \sqrt{1-\rho^2}\,\dd W^{\perp}(t))
\end{equation}% 14:00 ho skippato tutto il discorso che ha fatto
where $W^{\Qmeas}\indep W^{\perp}$. The parameter $\eta$ is called \emph{volatility of volatility parameter} and the parameter $\rho$ is called \emph{correlation parameter}. One interesting feature of this model is that the quadratic covariation between the asset returns and the instantaneous variance we get
\begin{equation*}
    \expval{\frac{\dd S}{S}, V} = V\eta\rho
\end{equation*}
which is positive or negative according to $\rho$ and can range from $-1$ to $+1$. Typically we have $\eta>0$, $V>0$ and $\rho<0$. The fact that $\rho<0$ is the explanation of the leverage effect of the volatility (when the asset value is growing the volatility is low and when the market crash the volatility is high). $\rho<0$ is also responsible for the negative slope of the smile effect, that is the skew. \\ % discorso 23:00 circa
The main advantage of this model is that it is analytically tractable, in the sense that we are able to do pricing even if we loose the log-normal property of the asset price. In other words, we are not able to compute prices as conditional expected value but through an analytical methodology which involves the Feynman-Kac formula and the Fourier transform. In order to introduce this methodology, we need some preliminary results on squared Bessel processes.

\subsection{Review of Bessel squared processes}
Let $V(t)$ be a time changed Bessel squared process (BESQ).
\begin{proposition}[Feller condition]
    If $2kV(\infty) \ge \eta^2$, then
    \begin{equation*}
        \text{Prob}(V(t)>0\,\forall t) = 1.
    \end{equation*}
\end{proposition}
This condition is very important because it guarantees the strict positivity of the volatility process.

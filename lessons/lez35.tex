\lesson{35}{29/05/2020} So, we need many parameters to describe the correlation between the vector Brownian motion driving the asset prices and the matrix Brownian motion driving the volatility. Luckily, the model has the following property.
\begin{proposition}
    The WASC model is affine if and only it the matrix $R_k$ has the following form:
    \begin{align}
        R_k = \mqty(0 & \cdots & 0 \\
                      \rho_1 & \cdots & \rho_n \\
                      0 & \cdots & 0) \leftarrow \text{$k$-th row}
    \end{align}
    with $\rho^T\rho \le 1$. So, we can write the Brownian motion $Z$ as the projection along the direction given by the vector $\rho$:
    \begin{align}
        \dd Z(t) = \dd W(t)\rho + \sqrt{1-\rho^T\rho}\,\dd B(t)
    \end{align}
    where $Z(t)$ and $B(t)$ are vectors and $W(t)$ is a matrix (independent of $B$).
\end{proposition}
So, we need only $n$ parameters to decribe the correlation.
\begin{proof}
    The starting point consists in taking the log-price. Adopting the notation
    \begin{align*}
        \vect{\Sigma_{ii}} = \mqty(\Sigma_{11} \\ \vdots \\ \Sigma_{nn}),
    \end{align*}
    we can write it as
    \begin{align*}
        \dd Y(t) = r\mathds{1}-\frac{1}{2}\vect{\Sigma_{ii}}\dd t + \sqrt{\Sigma}\,\dd Z(t).
    \end{align*}
    The idea is to show when $\expval{Y_h, \Sigma_{ij}}$ is affine in $\Sigma$ (if this is true then the whole infinitesimal generator will be affine). By Itô we get
    \begin{align*} % small signa = sqrt of the Wishart matrix
        \dd\expval{Y_h, \Sigma_{ij}} &= \sum_{k=1}^n \left(\sigma_{hk}\,\dd Z_k\right) \dd\Sigma_{ij} \\
        &=
        \sum_{k=1}^n \sigma_{hk} \Tr(R_k\,\dd W(t)^T)\left(\sum_{l,m=1}^n\sigma_{im}\,\dd W_{ml}Q_{lj}+\sigma_{jm}\,\dd W_{ml}Q_{li}\right) \\
        &=
        \sum_{k=1}^n \sigma_{hk} \sum_{p,q=1}^n R_k^{pq}\,\dd W_{pq}(t)\left(\sum_{l,m=1}^n\sigma_{im}\,\dd W_{ml}Q_{lj}+\sigma_{jm}\,\dd W_{ml}Q_{li}\right).
    \end{align*}
    Now, let's exploit the information that we have on the matrix Brownian motion $W$. Since all the elements are independent we have
    \begin{align*}
        \dd\expval{W_{pq},W_{ml}}_t = \delta_{pm}\delta_{ql}\,\dd t
    \end{align*}
    As a consequence, we get
    \begin{align*}
        \dd\expval{Y_h, \Sigma_{ij}} &= \sum_{m,l,k=1}^n \sigma_{hk}R_k^{ml}\left(\sigma_{im}Q_{lj} + \sigma_{jm}Q_{li}\right)\dd t \\
        &=
        \sum_{m,l,k=1}^n \left(\sigma_{hk}R_k^{ml}\sigma_{im}Q_{lj}+\sigma_{hk}R_k^{ml}\sigma_{jm}Q_{li}\right)\dd t
    \end{align*} %The only possibility to recognize the developement of the square of the matrix $\sigma$ is
    We would like to recognize the product $\sigma_{ij}\sigma_{ml}$ in such a way that by summing up over all the indices it leads to $\Sigma_{pq}$. The only possibility is to put $R_k^{ml}=0$ for all $k\ne m$ and $\forall l = 1,\dots,n$. We end up with the desired condition:
    \begin{itemize}
        \item $R_k$ must have all the elements equal to zero except for the $k$-th row;
        \item The rows of $R_k$ have the same elements for all $k$, so $\rho$ does not depend on $k$.
    \end{itemize}
    Therefore
    \begin{align*}
        \dd\expval{Y_h, \Sigma_{ij}} &= \sum_{m,l,k=1}^n \left(\sigma_{hk}\sigma_{im}Q_{lj}+\sigma_{hk}\sigma_{jm}Q_{li}\right)\rho_l\dd t \\
        \intertext{Fixing $l$ and summing over $m$ and $k$ we recognize the developement of the matrix $\Sigma_{hi}$ and $\Sigma_{hj}$:}
        &=
        \sum_{l=1}^n \left(\Sigma_{hi}Q_{lj} + \Sigma_{hj}Q_{li}\right)\rho_l\dd t \\
        &=
        \left(\sum_{hi}\Tr(R_jQ) + \Sigma_{hj}\Tr(R_iQ)\right)\dd t
    \end{align*}
    We end up with a linear combination in $\Sigma$, so the model is affine.
\end{proof}

\subsection{Pricing in a multiple asset model}
Given a payoff $F(Y(T),T)\in L^2(\mathcal{F}(T))$, its initial price is given by
\begin{align}
    \notag price_0 &= e^{-rT}\mathbb{E}^{\Qmeas}[F(Y(T),T)] \\
    &=
    \notag e^{-rT}\frac{1}{(2\pi)^n}\mathbb{E}^{\Qmeas}\left[\int_{\mathcal{Z}}e^{-i\expval{Z,Y}}\left(\int_{\mathbb{R}^n}e^{i\expval{Z,Y}}F(y,T)\,\dd y\right)\dd Z\right] \\
    &=
    \notag e^{-rT}\frac{1}{(2\pi)^n}\int_{\mathcal{Z}}\mathbb{E}^{\Qmeas}\left[e^{-i\expval{Z,Y}}\right]\left(\int_{\mathbb{R}}e^{i\expval{Z,Y}}F(y,T)\,\dd y\right)\dd Z \\
    \intertext{Once again we have that the price of a contingent claim can be written as the numerical integral of a product of two elements: the first one being the characteristic function of the asset prices (or log-prices) and the second being the characteristic function of the particular payoff we are interested in. The only difference is that now we are reasoning in multiple dimensions, where we can introduce the following notation:}
    &=
    \notag \frac{1}{(2\pi)^n}\int_{\mathcal{Z}}\psi_{Y(0),\Sigma(0)}(-iz,0,0,T)\hat{F}(Z)\,\dd Z
\end{align}
where
\begin{align}\label{ftofpayoff}
    \hat{F}(Z) = \int_{\mathbb{R}}e^{i\expval{Z,Y}}F(Y(T),T)\,\dd Y(T)
\end{align}
is the Fourier transform of the payoff and
\begin{align}
    \psi_{Y(t),\Sigma(t)}(\gamma,\Gamma,t,\tau) = e^{-r\tau} \expect\left[e^{\expval{\gamma,Y(t+\tau)}+\Tr[\Gamma \Sigma(t+\tau)]}\right]
\end{align}
This is the general Fourier transform: if we want to price calls or put we can skip the second term of the exponential whereas if we are interested in pricing volatility models we can skip the first term (in fact in \eqref{ftofpayoff} we set $\Gamma = 0$). By Feynman-Kac we can write
\begin{align}
    \pdv{\psi_{Y,\Sigma}}{\tau} = (\mathcal{A}_{Y,\Sigma}-r)\psi
\end{align}
\begin{proposition}
    The infinitesimal generator of the WASC model is given by
    \begin{align}
        \notag\mathcal{A}_{Y,\Sigma} &= \underbrace{\Tr((\beta Q^TQ + M\Sigma + \Sigma M^T)D + 2\Sigma DQ^TQD)}_{Wishart} + \\
        &\qquad\qquad
        + \underbrace{\nabla_Y\left(r\mathds{1}-\frac{1}{2}\vect{\Sigma_{ii}}\right) + \frac{1}{2}\nabla_Y\Sigma\nabla_Y^T}_{log-price\, Y} + 2\Tr(DQ^T\rho\nabla_Y\Sigma)
    \end{align}
    where $\rho$ is a column vector and $\nabla_Y$ is a row vector, so that $\rho\nabla_Y$ is a $n\times n$ matrix.
\end{proposition}% fine prima parte
\begin{proof}
    We altready know the infinitesimal generator of the Wishart process, so let's look for the other two. For the vector $Y(t)$ the infinitesimal generator is given by the Itô formula:
    \begin{align*}
        \mathcal{A}_Y = \sum_{i=1}^n\left(r-\frac{1}{2}\Sigma_{ii}\right)\pdv{}{y_i}+\frac{1}{2}\sum_{i,j=1}^n\Sigma_{ij}\pdv{}{y_i}{y_j}
    \end{align*}
    In terms of the covariation, we can write
    \begin{align*}
        \mathcal{A}_{\expval{Y,\Sigma}} &= 2\sum_{i=1}^n\Tr(\Sigma R_iQD)\pdv{}{y_i} \\
        &=
        2\Tr[\left(\sum_{i=1}^nR_i\pdv{}{y_i}\right)QD\Sigma] \\
        &=
        2
        \Tr\left[\mqty(\rho_1\pdv{}{y_1} & \cdots & \rho_n\pdv{}{y_1} \\
                       \vdots            & \ddots & \vdots \\
                       \rho_1\pdv{}{y_n} & \cdots & \rho_n\pdv{}{y_n})\right] \\
        &=
        2\Tr[\Sigma\nabla_y^T\rho^TQD]
    \end{align*}
\end{proof}
\begin{proposition}
    The Laplace transform of the joint process is given by
    \begin{align}
        \psi_{Y(t),\Sigma(t)}(\gamma,\Gamma,t,\tau) = e^{-r\tau} \exp{\Tr(A(\tau)\Sigma(t))+\gamma^TY(t)+c(\tau)}
    \end{align}
    where
    \begin{align}
        A(\tau) = (\Gamma A^1_2(\tau)+A_2^2(\tau))^{-1}(\Gamma A^1_1(\tau)+A_1^2(\tau))
    \end{align}
    is the solution of the Riccati equation and
    \begin{align}
        \mqty(A_1^1 & A_1^2 \\ A_2^1 & A_2^2) = \exp{\tau\mqty(M+Q^T\rho\gamma^T & -2Q^TQ \\ \tfrac{1}{2}\left(\gamma\gamma^T-\sum_{i=1}^nY_ie_{ii}\right) & -(M^T+\gamma\rho^TQ))},
    \end{align}
    \begin{align}
        c(\tau) = -\frac{\beta}{2}\Tr[\ln(\Gamma A_2^1(\tau)+A_2^2(\tau)) + \tau M^T + \tau\gamma\rho^TQ] + \tau r (\gamma^T\mathds{1}-1).
    \end{align}
\end{proposition}
\begin{proof}
    Putting the candidate into the PDE we get the Riccati equation
    \begin{align*}
        \begin{cases}
            \dot{A}(\tau) = AM + M^TA - \frac{1}{2}\sum_{i=1}^n \gamma_ie_{ii}+ 2AQ^TQA + \frac{1}{2}\gamma\gamma^T + AQ^T\rho\gamma^T + \gamma\rho^TQA \\
            A(0) = \Gamma
        \end{cases}
    \end{align*}
    together with
    \begin{align*}
        \begin{cases}
            \dot{c}(\tau) = \Tr(r\mathds{1}\gamma^T + \beta Q^TQA) - r \\
            c(0) = 0
        \end{cases}
    \end{align*}
    The linearization procedure is standard (express $A$ as a ratio involving the derivative of the denominator, ecc.). \colorbox{cyan}{Homework}
\end{proof}
Now we would like to see which are the other advantages of introducing a Wishart model instead of a multiple Heston. We have already seen that the Wishart process allows to consider stochastic correlations between the assets. Now we analyze what happens in the skew of the vanillas. If we are interested in pricing a call option on the first asset, in principle we can forget about the other assets. However we will see that in this model all the assets communicate with each other, because in the matrix $\sigma$ there is a ``cocktail" involving all the volatility parameters.

\subsection{Leverage in single name vanillas}
If we consider the correlation between driving the first (log-)asset price and its volatility:
\begin{align}
    \corr_t(\text{Noise}(Y_1),\text{Noise}(\text{Vol}(S_1))) &= \frac{\expval{Y_1,\Sigma_{11}}_t}{\sqrt{\Sigma_{11}(t)}\sqrt{\expval{\Sigma_{11}}_t}} \\
    &=
    \frac{2\Sigma_{11}\Tr(R_1Q)}{\sqrt{\Sigma_{11}(t)}\sqrt{4\Sigma_{11}(Q_{11}^2+Q_{21}^2)}} \\
    &=
    \frac{\Tr(R_1Q)}{\sqrt{Q_{11}^2+Q_{21}^2}} \\
    \intertext{Recall that $R_1 = \smqty(\rho_1 & \rho_2 \\ 0 & 0)$:}
    &=
    \frac{Q_{11}\rho_1+Q_{21}\rho_2}{\sqrt{Q_{11}^2+Q_{21}^2}}
\end{align}
So, we have a constant skew, like in the classic Heston model. This is due to the fact that we decided not to adopt the richest parametrization of the WASC model (which considers the trace for any asset).
\begin{remark}
    If the matrix $Q$ is diagonal
    \begin{align*}
        \corr_t(\text{Noise}(Y_1),\text{Noise}(\text{Vol}(S_1))) &= \rho_1
        \corr_t(\text{Noise}(Y_2),\text{Noise}(\text{Vol}(S_2))) &= \rho_2
    \end{align*}
    as in the classic Heston model. If $Q$ is not diagonal, the skew of the implied volatility on asset 1 depends on the volatility of asset 2.
\end{remark}
In some sense, in this framework vanillas are ``basket options", in the sense that there are dependences between assets through their volatilities.

\subsection{Asymmetric conditional correlation effect}
In the stochastic volatility framework for a single asset, if the market rises up we observe that the volatility $\sigma$ decreases while if the market crushes $\sigma$ increases. In the multiple assets case:
\begin{itemize}
    \item if the market rises up, the correlation $\rho$ between the assets decreases, in the sense that there is more diversification in the market;
    \item if the market crushes, the correlation between the assets increases. There is a contagion effect, in the sense that the credit risk transfers from one asset to another. This means that there will no more diversification in the market (all the assets behave more or less in the same way).
\end{itemize}
We would like to see if there exists a model that can describe this effect. In the WASC model we can find a closed form expression of the covariation between the asset $Y_i$ and the cross correlation
\begin{align}
    \rho_{Y_1,Y_2}(t) = \rho_{12}(t) = \frac{\Sigma_{12}(t)}{\sqrt{\Sigma_{11}(t)\Sigma_{22}(t)}}.
\end{align}
\begin{proposition}
    The covariation between the $i$-th asset price and the cross correlation $\rho_{12}$ is given by
    \begin{align}
        \dd\expval{Y_i,\rho_{12}} = \frac{\sqrt{\Sigma_{ii}}}{\Sigma_{jj}}(1-\rho_{12}^2)\Tr(R_jQ)\dd t
    \end{align}
    (where in our case $i,j=1,2$).
\end{proposition}
\begin{proof}
    By definition,
    \begin{align*}
        \rho_{12}\sqrt{\Sigma_{11}\Sigma_{22}} = \Sigma_{12}.
    \end{align*}
    By differentiating, we get
    \begin{align*}
        \dd\Sigma_{12} = \sqrt{\Sigma_{11}\Sigma_{22}}\,\dd \rho_{12} + \rho_{12}\,\dd(\sqrt{\Sigma_{11}\Sigma_{22}}) + (\dots)\dd t.
    \end{align*}
    By Itô
    \begin{align*}
        \dd(\sqrt{\Sigma_{11}\Sigma_{22}}) = \frac{1}{2}\sqrt{\frac{\Sigma_{22}}{\Sigma_{11}}}\,\dd\Sigma_{11} + \frac{1}{2}\sqrt{\frac{\Sigma_{11}}{\Sigma_{22}}}\,\dd\Sigma_{22} + (\dots)\dd t.
    \end{align*}
    So, we can write
    \begin{align*}
          \dd\rho_{12} = \frac{1}{\sqrt{\Sigma_{11}\Sigma_{22}}}\,\dd \Sigma_{12} - \frac{1}{2}\rho_{12}\left(
          \frac{1}{\Sigma_{11}}\,\dd\Sigma_{11} +
          \frac{1}{\Sigma_{22}}\,\dd\Sigma_{22}\right) + (\dots)\dd t.
    \end{align*}
    Then
    \begin{align*}
        \dd\expval{Y_1,\rho_{12}}_t &= \frac{1}{\sqrt{\Sigma_{11}\Sigma_{22}}}\,\dd\expval{Y_1,\Sigma_{12}} - \frac{1}{2}\rho_{12} \left(
        \frac{1}{\Sigma_{11}}\,\dd\expval{Y_1,\Sigma_{11}} + \frac{1}{\Sigma_{22}}\,\dd\expval{Y_1,\Sigma_{22}}
        \right) \\
        &=
        \frac{1}{\sqrt{\Sigma_{11}\Sigma_{22}}}\left(\Sigma_{11}\Tr(R_2Q)+\Sigma_{12}\Tr(R_1Q)\right)\dd t - \\
        &\qquad\qquad\qquad\qquad\qquad\qquad - \rho_{12}\left(\Tr(R_1Q) + \frac{\Sigma_{12}}{\Sigma_{22}}\Tr(R_2Q)\right)\dd t \\
        &=
        \Tr(R_2Q)\left(\frac{\Sigma_{11}}{\sqrt{\Sigma_{11}\Sigma_{22}}}-\rho_{12}\frac{\Sigma_{12}}{\Sigma_{22}}\right)\dd t \\
        &=
        \Tr(R_2Q)\frac{\sqrt{\Sigma_{11}}}{\Sigma_{22}}(1-\rho_{12}). % questa dimostrazione ha un errore perchè dovrebbe essere rho_12 al quadrato
    \end{align*}
\end{proof}
We have that $(1-\rho_{12})\ge0$ and $\tfrac{\sqrt{\Sigma_{ii}}}{\Sigma_{jj}}\ge0$, so
\begin{equation*}
    \frac{\sqrt{\Sigma_{ii}}}{\Sigma_{jj}}(1-\rho_{12}^2) \ge 0
\end{equation*}
Conversely, the term $\Tr(R_jQ)$ is related to the skew of the vanilla on the asset $j$, which typically is negative:
\begin{align*}
    \Tr(R_jQ) \le 0 \qquad\Rightarrow\qquad \dd\expval{Y_i,\rho_{12}} \le 0
\end{align*}
The correlation of every combination of the assets with the cross correlation among the assets is typically negative. So, if we define and index as the weighted sum of the log-assets
\begin{equation*}
    \text{Index} = \sum_i \alpha_iY_i
\end{equation*}
when this index decreases, due to the fact that $\expval{\text{Index},\rho_{ij}}<0$, the cross correlation $\rho_{12}$ increases. This is the meaning of the contagion effect.

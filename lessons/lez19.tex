\section{Quantity-adjusted options}\lesson{18}{22/04/2020} % Bjork pag. 248
Quantity-adjusted options, also called \emph{quanto} options, are cash settled, cross currency derivatives where the underlying asset is denominated in a currency different from the one in which the option is settled. In other words, quanto options are derivatives where the payoff is defined by variables associated with one currency but is paid in another currency. \\
For example, the transfer between \EUR{1} and 1\,\$ is given through an \emph{exchange rate}:
\begin{equation}
    X(t)^{\text{EURUSD}} = \dfrac{\text{\EUR{1}}}{1\,\$}
\end{equation}
which is usually given using the \emph{FORDOM} (foreign-domestic currency) notation. If we consider a third additional currency, for example the Japanese Yen \textyen, in order to be consistent it must be that if $X_1, X_2$ are two exchange rate, then $X_3 = X_1X_2$. So, the product of two exchange rates is an exchange rate.\\
In general, the interest rates in different countries are different, so  -- at least in principle -- there can be a potential arbitrage opportunity. This problem is only apparent because as soon as we want to exploit this arbitrage we have to consider the currency conversion, which works through a stochastic quantity, i.e. the interest rate. \\
As starting point, we assume that the evolution of the exchange rate between the two currencies is given by a B\&S dynamics under the subjective probability measure $\Pmeas$:
\begin{align}
    \frac{\dd X(t)}{X(t)} = \mu_X\,\dd t + \sigma_X\,\dd W^{\Pmeas}(t)
\end{align}
This is a good candidate, in fact the products of two Brownian motions is itself a Brownian motion. The evolution of the corresponding riskless asset in the domestic currency is given by:
\begin{align}\label{dd}
    \frac{\dd B_d}{B_d} = r_d\,\dd t
\end{align}
Similarly, the riskless asset in the foreign currency evolves according to the equation:
\begin{equation}\label{f}
    \frac{\dd B_f}{B_f} = r_f\,\dd t
\end{equation}
We already know the solutions of \eqref{dd} and \eqref{f}:
\begin{equation}
    B_d(t) = e^{r_d t}, \qquad B_f(t) = e^{r_f t}
\end{equation}
Here we assume that $r_d, r_f$ are constant and that if we buy foreign currency, this will evolve according $r_f$\footnote{This is not always the case. For example, if we buy dollars in the euro-zone the evolution of this EUR-dollar is different from the one of USD-dollar. Anyway, this is a reasonable assumption.}. \\
We want to price options on the exchange rate at maturity $X_T$. These are not properly quanto options, because we need to introduce an underlying denominated in a different currency. For example, consider the call option that allows to buy 1\,\$ at price \EUR{$K$}, which has payoff $(X_T-K)^+$. According to the risk neutral methodology the price is given by:
\begin{align*}
    price_t = e^{-r(T-t)}\expect_t[(X_T-K)^+]
\end{align*}
But we are not able to define the risk neutral measure $\Qmeas$ if there is no underlying. In fact, in order to find $\Qmeas^d$ we have to impose that whatever discounted asset in our country is a martingale, i.e. it evolves with a drift $r_d$. \\
Which are the domestic assets? Let's consider a stylized economy in which we have the foreign and domestic saving accounts and we consider the transfers between the two. In this case there are two assets:
\begin{itemize}
    \item $B_d$, in fact its drift is $r_d$;
    \item $B_fX = \Tilde{B}_f$, i.e. the foreign riskless asset converted into the domestic currency.  
\end{itemize}
By Itô, we have that:
\begin{align}\label{al1}
    \notag\dd\Tilde{B}_f &= \dd(B_fX) = (\dd B_f)X + B_f\,\dd X \\
    &= 
    \notag\Tilde{B}_f r_f\, \dd t + \Tilde{B}_f(\mu_X\,\dd t + \sigma_X\,\dd W^{\Pmeas}) \\
    &= 
    \Tilde{B}_f(\mu_X + r_f)\,\dd t + \sigma_X\,\dd W^{\Pmeas}
\end{align}
This is the general dynamics of a foreign riskless asset converted in the domestic currency. Now we want that 
\begin{equation*}
    \mu_X + r_f = r_d
\end{equation*}
so we perform the same trick we used when we applied the martingale approach to the B\&S model, that is we absorb the excessive drift into the Brownian motion:
\begin{equation}\label{al2}
    \eqref{al1} = \Tilde{B}_f\left(r_d\,\dd t + \sigma_X\left(\dd W^{\Pmeas} + \dfrac{\mu_X + r_f-r_d}{\sigma_X}\,\dd t\right)\right)
\end{equation}
If we define the change of probability measure given by the Radon-Nikodym derivative\footnote{$\mathcal{E}$ is the so called \href{https://en.wikipedia.org/wiki/Dol\%C3\%A9ans-Dade\_exponential}{Doléans-Dade exponential}.}:
\begin{align}
    \notag\eval{\frac{\dd \Qmeas^d}{\dd \Pmeas}}_t &= \mathcal{E}\left(\dfrac{\mu_X + r_f-r_d}{\sigma_X}\right)_t \\
    &=
    \notag\exp{\int_0^t \left(\dfrac{\mu_X + r_f-r_d}{\sigma_X}\right)\,\dd W^{\Pmeas} - \int_0^t \left(\dfrac{\mu_X + r_f-r_d}{\sigma_X}\right)^2\,\dd s} \\
    &\sim 
    \exp{\alpha W(t) - \frac{1}{2}\sigma^2 t} = \text{true martingale}
\end{align}
we can apply the Girsanov theorem:
\begin{equation*}
    \dd W^{\Pmeas} + \dfrac{\mu_X + r_f-r_d}{\sigma_X}\,\dd t = \dd W^{\Qmeas^d}
\end{equation*}
So:
\begin{align}
    \eqref{al2} &= \Tilde{B}_f(r_d\,\dd t + \sigma_X\dd W^{\Qmeas^d})
\end{align}
So, the dynamics of $X_t$ under the domestic risk neutral pricing probability measure $\Qmeas^d$ is given by:
\begin{equation}\label{dynX}
    \dfrac{\dd X_t}{X} = (r_d - r_f)\dd t + \sigma_X\,\dd W^{\Qmeas^d}
\end{equation}
In a sense, the short foreign interest rate $r_f$ plays the role of a dividend.
\begin{remark}
    Notice that the exchange rate $X/B^d$, i.e. the discounted $X$, is not a $\Qmeas^d$-martingale. This is confirmed by the fact that there is an additional drift $r_f$.
\end{remark} % fine prima parte
Now we are ready to price an exchange option.
\begin{proposition}
    The price at time $t$ of whatever payoff written on the exchange rate at maturity $T$ is given by:
    \begin{itemize}
    \item the discounted value under the domestic interest rate of the conditional risk neutral expectation of the payoff:
    \begin{equation}
    price_t = e^{-r_d(T-t)}\mathbb{E}^{\Qmeas^d}_{t,X}[\pay(X_T)]
    \end{equation}
    where the dynamics of $X_t$ is given by \eqref{dynX};
    \item the solution:
    \begin{equation}
        price_t = F(t.X_t) 
    \end{equation}
    of the PDE:
    \begin{equation}
        \begin{cases}
        \pdv{F}{t} + (r_d-r_f)X\pdv{F}{X} + \frac{1}{2}X^2\sigma_X^2\pdv[2]{F}{X} - r_dF = 0\\
        F(X,T) = \pay(X)
        \end{cases}
    \end{equation}
    \item the solution \begin{equation}
        F(t,X) = F_0(t,Xe^{-r_f(T-t)}) 
    \end{equation}
    where $F_0$ is the solution corresponding to no dividends, which obeys to the equation:
    \begin{equation}
        \frac{\dd X}{X} = r_d\dd t + \sigma_X\dd W^{\Qmeas}
    \end{equation}
    \end{itemize}
\end{proposition}
\begin{example}{Call option on the exchange rate}{}{}
    The price of a call option on $X$ is given by:
   \begin{align}
       price_t(\call\text{ on } X) = X_t e^{-r_f(T-t)}\Phi(d_1) - e^{-r_d(T-t)}K\Phi(d_2)
   \end{align}
   where
   \begin{equation}
       d_1 = \frac{1}{\sigma_X\sqrt{T-t}}\left(\ln\frac{X_te^{-r_f(T-t)}}{K}+\left(r_d+\frac{1}{2}\sigma_X^2\right)(T-t)\right)
   \end{equation}
   \begin{equation}
       d_2 = \frac{1}{\sigma_X\sqrt{T-t}}\left(\ln\frac{X_te^{-r_f(T-t)}}{K}+\left(r_d-\frac{1}{2}\sigma_X^2\right)(T-t)\right)
   \end{equation}
\end{example}
Yet, we are not ready to price quanto options. We need to do an upgrade, introducing in the market some risky assets. The starting point is that the dynamics under $\Pmeas$ is given by
\begin{equation}
    \frac{\dd X}{X} = \mu_X\,\dd t + \sigma_X\cdot\dd W^{\Pmeas}
\end{equation}
for the exchange rate, by
\begin{equation}\label{Sd}
    \frac{\dd S_d}{S_d} = \mu_d\,\dd t + \sigma_d\cdot\dd W^{\Pmeas}
\end{equation}\label{Sf}
for the domestic asset and by
\begin{equation}
    \frac{\dd S_f}{S_f} = \mu_f\,\dd t + \sigma_f\cdot\dd W^{\Pmeas}
\end{equation}
where we used the scalar product $(\cdot)$ because $\sigma$ is a row vector and $\dd W$ is a column vector.\\
The volatility matrix of our market is given by
\begin{equation}
    \sigma = 
    \begin{pmatrix}
    \sigma_X \\ \sigma_d \\ \sigma_f
    \end{pmatrix} =
    \begin{pmatrix}
    \sigma_{X_1} & \sigma_{X_2} & \sigma_{X_3} \\
    \sigma_{d_1} & \sigma_{d_2} & \sigma_{d_3} \\
    \sigma_{f_1} & \sigma_{f_2} & \sigma_{f_3} 
    \end{pmatrix}
\end{equation}
We also add a domestic and a foreign riskless assets which evolve according to the equations:
\begin{align}
    \frac{\dd B_d}{B_d} &= r_d\,\dd t \\
    \frac{\dd B_f}{B_f} &= r_f\,\dd t
\end{align}
Now we can consider different problems involving different contract. For example:
\begin{itemize}
    \item pricing in euros a call on $S_f$ in dollars with $K$ in dollars, which has payoff $(S_f(T)-K)^+X_T^{\text{USDEUR}}$ (not a real quanto option);
    \item pricing in euros a call on $S_f$ in dollars with $K$ in euros, which has payoff $(S_fX_T-K)^+$ (real quanto option).
\end{itemize}
Now we have to characterize the probability measure $\Qmeas^d$ under which we price this payoffs. Again, we impose that under $\Qmeas^d$ all the domestic assets discounted by the domestic interest rate are a $\Qmeas^d$-martingale, i.e. the domestic assets evolve with drift $r_d$. Our domestic assets are $B_d, S_d, \Tilde{S}_f = S_fX_t$ and $\Tilde{B}_f = B_fX_t$, so this quantities have -- once discounted by $B_d$ -- must be $\Qmeas^d$-martingales.\\
Let's start from $S_d$ and rewrite \eqref{Sd} in such a way that the drift is $r_d$:
\begin{align}
    \frac{\dd S_d}{S_d} = r_d\,\dd t + \sigma_d\left(\dd W^{\Pmeas} + \frac{\mu_d-r_d}{``\sigma_d"}\,\dd t\right)
\end{align}
This is only a formal expression because, since $\sigma_d$ is a row vector, $\nicefrac{1}{\sigma_d}$ is not well defined.\\
Then we have:
\begin{align}
    \notag\frac{\dd\Tilde{S}_f}{\Tilde{S}_f} &= (\mu_f+\mu_X+\sigma_f+\sigma_T)\,\dd t + (\sigma_f + \sigma_d)\dd W^{\Pmeas} \\
    &=
    r_d\,\dd t + (\sigma_f+\sigma_X)\left(\dd W^{\Pmeas}+\dfrac{\mu_f + \mu_X + \sigma_f\sigma_X^T - r_d}{``\sigma_f\sigma_X"}\,\dd t\right)
\end{align}
which again is just a formal expression. Finally, we can do the same thing for the foreign riskless asset denominated in the domestic currency:
\begin{align}
    \notag\frac{\dd\Tilde{B}_f}{\Tilde{B}_f} &= (\mu_f+\mu_X)\,\dd t + \sigma_X\dd W^{\Pmeas} \\
    &=
    r_d\,\dd t + \sigma_X\left(\dd W^{\Pmeas}+\dfrac{\mu_X + r_f - r_d}{``\sigma_X"}\,\dd t\right) 
\end{align}
Provided that the Girsanov Theorem can be applied to a 3-dimensional geometric Brownian motion $W^{\Pmeas}$ we have to find the corresponding Radon-Nikodym derivative 
\begin{equation} % 24:00
    \eval{\dfrac{\dd\Qmeas^d}{\dd\Pmeas}}_t = \colorbox{cyan}{homework}
\end{equation}
Now, for example, consider the problem of pricing in euros a call on $S_f$ in dollars with $K$ in dollars, which has payoff $(S_f(T)-K)^+X_T^{\text{USDEUR}}$. We can solve this problem by pricing directly in the foreign country, that is under $\Qmeas^f$, using the B\&S model and then convert the result using the exchange rate: 
\begin{equation}
    price = X_t(S_f(t)\Phi(d_1)-e^{-r_f(T-t)}K\Phi(d_2))
\end{equation}
where 
\begin{equation}
    d_1 = \frac{1}{\norm{\sigma_f}\sqrt{T-t}}\left(\ln\frac{S_f(t)}{K}+\left(r_f+\frac{1}{2}\norm{\sigma_f}^2\right)(T-t)\right)
\end{equation}
Again, we see that this is not a real quanto problem but just a matter of conversion.
Let's consider the real quanto option problem, i.e. pricing in euros a call on $S_f$ in dollars with $K$ in euros, which has payoff $(S_f X_T-K)^+$. This is a call written on $\Tilde{S}_f$ so we can apply the $\Qmeas^d$-pricing methodology on it. So, the price of this quanto option is:
\begin{equation}
    price_t((S_f X_T-K)^+) = \Tilde{S}_f(t)\Phi(d_1)-e^{-r_d(T-t)}K\Phi(d_2)
\end{equation}
with
\begin{equation}
    d_1 = \frac{1}{\norm{\sigma_X+\sigma_f}\sqrt{T-t}}\left(\ln\frac{\Tilde{S}_f(t)}{K}+\left(r_d+\frac{1}{2}\norm{\sigma_X+\sigma_f}^2\right)(T-t)\right)
\end{equation}
\begin{remark}
    $\Qmeas^d \ne \Qmeas^f$ (Siegel paradox). This happens because under $\Qmeas^d$ we impose that everything in the domestic market -- after being discounted by the domestic interest rate -- is a martingale while under $\Qmeas^f$ we impose that all the foreign assets, including the domestic ones converted in the foreign currency, will be martingales -- once discounted by the foreign riskless asset. This leads to different dynamics. In particular, the exchange rate under $\Qmeas^d$ evolves according to the equation:
    \begin{equation}
       \frac{\dd X}{X} = (r_d-r_f)\,\dd t + \sigma_X\,\dd W^{\Qmeas^d}  
    \end{equation}
    Consider the euro-zone and the dollar-zone. If the exchange rate between euros and dollars is $X$ then the one from dollars to euros must be $\nicefrac{1}{X}$. But the dynamics under $\nicefrac{1}{X}$ is given by 
    \begin{equation}
        \frac{\dd\frac{1}{X}}{\frac{1}{X}} = (r_d-r_f\hlc{mypink}{\norm{\sigma_X}^2})\,\dd t + \sigma_X\,\dd W^{\Qmeas^d}  
    \end{equation}
    so there is an additional term, essentially the Itô term, which is responsible for the fact that the two measures are not the same. This is due to the randomness of the exchange rate.  
\end{remark}
\begin{example}{Quanto exchange call}{}{}
    Price the exchange quanto option, which has payoff:
    \begin{equation*}
        (X_TS_f(T)-S_d(T))^+ = (\Tilde{S}_f(T)-S_d(T))^+
    \end{equation*}
    \colorbox{cyan}{Homework}.
\end{example}




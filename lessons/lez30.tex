\section{Implementation of the Heston model}\lesson{30}{20/05/2020}
\colorbox{cyan}{To do.} Also see \href{http://www.optioncity.net/pubs/Ch2Excerpt.pdf}{this article by Lewis.}
% vedi appunti che ho preso sul foglio di carta
% lez30 dal 00:00 a 16:50

\section{Multi-dimensional Riccati equation}
In the next sections we will consider the following situations:
\begin{itemize}
    \item increase the dimension of the volatility factors. In fact, explaining the whole volatility surface in terms of the spot instantaneous variance (which is a 1-dimensional stochastic process) is quite reductive. In order to get a richer explanation we have to introduce more factors;
    \item If we introduce more factors we have to calibrate according to maket data (usually option prices), which can be very difficult and expensive. So, we must try to be parsimonious.
\end{itemize}
So, we have a trade-off between complexity and parsimony/analytical tractability (at least keep the property of the Heston model, i.e. the possibility to compute the characteristic function). The computation of the characteristic function for the payoff is not a problem, in fact the problem is to find the characteristic function of the model. In order to do that, in the classic Heston model we have to solve the Riccati equation, which is done transforming the quadratic first order ODE into a linear second order ODE. This technique works well in the one-dimensional case but not in the multi-dimensional one.  \\
There are many ways to extend the one-dimensional case, but the most natural just introduce a vector of volatility factors. For example, in order to deal with the two-dimensional Heston model we introduce a 2-d vector with two volatility factors, one for the short term implied volatility surface and one for the long term volatility smile. In this way we can separately manage the smile and the skew of the short term and long term expiring options. However, this is not te best procedure to follow if we want to keep a certain level of analytical tractability and moreover it introduces some intrinsic constraints. In order to understand which is the right technology to exploit and which are the potential issues in the naive generalization, we proceed step by step.\\
First, we need a technique to solve the Riccati ODE
\begin{equation*}
    \begin{cases}
        \dot{B} = \frac{1}{2}\xi^2B^2 - (K-\rho\xi i z)B - \frac{1}{2}(iz+z^2) \\
        B(0) = 0.
    \end{cases}
\end{equation*}
The linearization technique says that -- instead of considering that the solution is given by the ratio of the derivative of a function and the function itself -- the solution is given by
\begin{equation}\label{bhf}
    B = H^{-1}F
\end{equation}
where $H \ne 0$ and $F$ are deterministic functions. Assume that $H$ and $F$ are both matrices. Then, if we muliply both size of \eqref{bhf} we get
\begin{equation*}
    HB = F
\end{equation*}
Then, if we differentiate, we get
\begin{equation*}
    \dot{H}B + H\dot{B} = \dot{F}
\end{equation*}
Now, let's introduce $\dot{B}$ as defined by the Riccati equation:
\begin{align*}
    \dot{H}B + \frac{1}{2}\xi^2B^2H - H(K-\rho\xi i z)B - \frac{1}{2}(iz+z^2)H = \dot{F}
\end{align*}
Rearranging:
\begin{equation*}
    \dot{F} - \dot{H}B = \frac{1}{2}\xi^2FB - (K-\rho\xi i z)F - \frac{1}{2}(iz+z^2)H
\end{equation*}
Now, let's consider the system obtained by identifying the coefficients of $B$ and the other (constant) terms:
\begin{equation*}
    \begin{cases}
        \dot{H} = -\frac{1}{2}\xi^2F \\
        \dot{F} = -(K-\rho\xi i z)F - \frac{1}{2}(iz+z^2)H
    \end{cases}
\end{equation*}
In this way we have transformed the quadratic first order ODE into a system of two linear ODEs, which can be rewritten as
\begin{equation}\label{HFlinearsys}
    \mqty(\dot{F}\\\dot{H}) = \mqty(-(K-\rho\xi iz) & - \frac{1}{2}(iz+z^2) \\ -\tfrac{1}{2}\xi^2 & 0) \mqty(F \\ H)
\end{equation}
So, if $F$ and $H$ are two functions that solve this linear system with the initial conditions
\begin{align}
    F(0) = 0, \qquad H(0) = 0
\end{align}
(such that $H_0^{-1}F_0=B_0=0$) then we can plug them into the Riccati equation. The solution of eq. \eqref{HFlinearsys} is given by
\begin{align}\label{HFlinearsysSol}
    \notag \mqty(F_{\tau} \\ H_{\tau}) &= \exp{\mqty(-(K-\rho\xi iz)\tau & - \frac{1}{2}(iz+z^2)\tau \\ -\tfrac{1}{2}\xi^2\tau & 0)}\mqty(0\\1) \\
    &=
    \mqty(A_{11}(\tau) & A_{12}(\tau) \\ A_{21}(\tau) & A_{22}(\tau))\mqty(0\\1) \\
    &=
    \mqty(A_{12}(\tau) \\ A_{21}(\tau))
\end{align}
Then, the solution of the Riccati equation at time $\tau$ will be
\begin{eqnarray}
    B(\tau) = A_{22}^{-1}(\tau) A_{12}(\tau)
\end{eqnarray}
which -- provided that we are able to compute the exponential of the matrix in \eqref{HFlinearsysSol} -- is completely explicit. % fine parte 1

\section{Implementation of the Heston model}\lesson{30}{20/05/2020}
To do. Also see \href{http://www.optioncity.net/pubs/Ch2Excerpt.pdf}{this article by Lewis}
% vedi appunti che ho preso sul foglio di carta
% lez30 dal 00:00 a 16:50

\section{Next steps}
In the next sections we will consider the following situations:
\begin{itemize}
    \item increase the dimension of the volatility factors. In fact, explaining the whole volatility surface in terms of the spot instantaneous variance (which is a 1-dimensional stochastic process) is quite reductive. In order to get a richer explanation we have to introduce more factors;
    \item If we introduce more factors we have to calibrate according to maket data (usually option prices), which can be very difficult and expensive. So, we must try to be parsimonious.
\end{itemize}
So, we have a trade-off between complexity and parsimony/analytical tractability (at least keep the property of the Heston model, i.e. the possibility to compute the characteristic function).

20:55

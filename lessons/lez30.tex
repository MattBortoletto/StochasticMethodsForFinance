\section{Implementation of the Heston model}
\colorbox{cyan}{To do.} Also see \href{http://www.optioncity.net/pubs/Ch2Excerpt.pdf}{this article by Lewis.}
% vedi appunti che ho preso sul foglio di carta
% lez30 dal 00:00 a 16:50

\section{Multi-dimensional Riccati equation}
In the next sections we will consider the following situations:
\begin{itemize}
    \item increase the dimension of the volatility factors. In fact, explaining the whole volatility surface in terms of the spot instantaneous variance (which is a 1-dimensional stochastic process) is quite reductive. In order to get a richer explanation we have to introduce more factors;
    \item If we introduce more factors we have to calibrate according to maket data (usually option prices), which can be very difficult and expensive. So, we must try to be parsimonious.
\end{itemize}
So, we have a trade-off between complexity and parsimony/analytical tractability (at least keep the property of the Heston model, i.e. the possibility to compute the characteristic function). The computation of the characteristic function for the payoff is not a problem, in fact the problem is to find the characteristic function of the model. In order to do that, in the classic Heston model we have to solve the Riccati equation, which is done transforming the quadratic first order ODE into a linear second order ODE. This technique works well in the one-dimensional case but not in the multi-dimensional one.  \\
There are many ways to extend the one-dimensional case, but the most natural just introduces a vector of volatility factors. For example, in order to deal with the two-dimensional Heston model we introduce a 2-d vector with two volatility factors, one for the short term implied volatility surface and one for the long term volatility smile. In this way we can separately manage the smile and the skew of the short term and long term expiring options. However, this is not te best procedure to follow if we want to keep a certain level of analytical tractability and moreover it introduces some intrinsic constraints. In order to understand which is the right technology to exploit and which are the potential issues in the naive generalization, we proceed step by step.\\
First, we need a technique to solve the Riccati ODE
\begin{equation*}
    \begin{cases}
        \dot{B} = \frac{1}{2}\xi^2B^2 - (K-\rho\xi i z)B - \frac{1}{2}(iz+z^2) \\
        B(0) = 0.
    \end{cases}
\end{equation*}
The linearization technique says that -- instead of considering that the solution is given by the ratio of the derivative of a function and the function itself -- the solution is given by
\begin{equation}\label{bhf}
    B = H^{-1}F
\end{equation}
where $H \ne 0$ and $F$ are deterministic functions. Assume that $H$ and $F$ are both matrices. Then, if we muliply by $H$ both sides of \eqref{bhf} we get
\begin{equation*}
    HB = F
\end{equation*}
Then, if we differentiate, we get
\begin{equation*}
    \dot{H}B + H\dot{B} = \dot{F}
\end{equation*}
Now, let's introduce $\dot{B}$ as defined by the Riccati equation:
\begin{align*}
    \dot{H}B + \frac{1}{2}\xi^2B^2H - H(K-\rho\xi i z)B - \frac{1}{2}(iz+z^2)H = \dot{F}
\end{align*}
Rearranging:
\begin{equation*}
    \dot{F} - \dot{H}B = \frac{1}{2}\xi^2FB - (K-\rho\xi i z)F - \frac{1}{2}(iz+z^2)H
\end{equation*}
Now, let's consider the system obtained by identifying the coefficients of $B$ and the other (constant) terms:
\begin{equation*}
    \begin{cases}
        \dot{H} = -\frac{1}{2}\xi^2F \\
        \dot{F} = -(K-\rho\xi i z)F - \frac{1}{2}(iz+z^2)H
    \end{cases}
\end{equation*}
In this way we have transformed the quadratic first order ODE into a system of two linear ODEs, which can be rewritten as
\begin{equation}\label{HFlinearsys}
    \mqty(\dot{F}\\\dot{H}) = \mqty(-(K-\rho\xi iz) & - \frac{1}{2}(iz+z^2) \\ -\tfrac{1}{2}\xi^2 & 0) \mqty(F \\ H)
\end{equation}
So, if $F$ and $H$ are two functions that solve this linear system with the initial conditions
\begin{align}
    F(0) = 0, \qquad H(0) = 0
\end{align}
(such that $H_0^{-1}F_0=B_0=0$) then we can plug them into the Riccati equation. The solution of eq. \eqref{HFlinearsys} is given by
\begin{align}\label{HFlinearsysSol}
    \mqty(F_{\tau} \\ H_{\tau}) &= \exp{\mqty(-(K-\rho\xi iz)\tau & - \frac{1}{2}(iz+z^2)\tau \\ -\tfrac{1}{2}\xi^2\tau & 0)}\mqty(0\\1) \\
    &=
    \notag\mqty(A_{11}(\tau) & A_{12}(\tau) \\ A_{21}(\tau) & A_{22}(\tau))\mqty(0\\1) \\
    &=
    \notag\mqty(A_{12}(\tau) \\ A_{21}(\tau))
\end{align}
Then, the solution of the Riccati equation at time $\tau$ will be
\begin{eqnarray}
    B(\tau) = A_{22}^{-1}(\tau) A_{12}(\tau)
\end{eqnarray}
which -- provided that we are able to compute the exponential of the matrix in \eqref{HFlinearsysSol} -- is completely explicit. % fine parte 1

\subsection{Matrix exponential}
Given a $d\times d$ matrix $A\in M(\mathbb{R}_d)$, its exponential is defined as
\begin{equation*}
    e^A = \mathds{1} + A + \frac{A^2}{2!} + \dots = \sum_{n=0}^{+\infty} \frac{A^n}{n!}.
\end{equation*}
If $A$ is symmetric, $A = A^T$, all the eigenvalues are real and we can diagonalize it: there exists a matrix $P$ such that $P^{-1}=P^T$ such that
\begin{equation*}
    A = PDP^{-1}
\end{equation*}
where
\begin{equation*}
    D = \mqty(\dmat{\lambda_1,\ddots,\lambda_d}).
\end{equation*}
In this case the exponential of $A$ is given by
\begin{align*}
    e^A &= \mathds{1} + PDP^{-1} + \frac{(PDP^{-1})^2}{2!} + \cdot \\
    &=
    PP^{-1} + PDP^{-1} + \frac{PD^2P^{-1}}{2!} + \cdot \\
    &=
    P\left(\mathds{1} + D + \frac{D^2}{2!} + \cdot\right)P^{-1} \\
    &=
    Pe^DP^{-1} \\
    &=
    P\mqty(\dmat{e^{\lambda_1},\ddots,e^{\lambda_d}})P^{-1}.
\end{align*}
\begin{remark}
    For a symmetric positive semi-definite matrix $A\in S^+_d$ we have that
    \begin{align*}
        \sqrt{A} &= \sqrt{PDP^{-1}} = \sqrt{P\sqrt{D}\sqrt{D}P^{-1}} \\
        &=
        \sqrt{P\sqrt{D}P^{-1}P\sqrt{D}P^{-1}} = \sqrt{(P\sqrt{D}P^{-1})^2} = P\sqrt{D}P^{-1}
    \end{align*}
    This is not the Cholesky decomposition $A=LU$. Here $\sqrt{A}$ is also in $S^+_d$.
\end{remark}
\begin{remark}
    In general, $\sqrt{A}$ may not exist or be not unique. For example, $A=\smqty(0 & 1 \\ 0 & 0)$ has no square root and $A = \smqty(33 & 24 \\ 48 & 57)$ has multiple square roots (for example $\smqty(1 & 4 \\ 8 & 5)$ or $\smqty(5 & 2 \\ 4 & 7)$). However, postive semi-definite matrices have only one positive semi-definite square root (\emph{principal square root}).
\end{remark}
In conclusion, care must be taken when applying to multiple arguments functions typically defined in $\mathbb{R}$.

\section{Multifactor stochastic volatility models}
\subsection{2-Heston model for one asset}
Consider an underlying that evolves under the risk neutral probability measure according to
\begin{equation}
    \frac{\dd S(t)}{S(t)} = r\,\dd t + \sqrt{V_1(t)}\,\dd Z_1(t) + \sqrt{V_2(t)}\,\dd Z_2(t)
\end{equation}
where $V_1$ and $V_2$ are respectively associated to the short term and the long term behavior of the volatility surface. The dynamics of $V_1$ and $V_2$ are given by the Heston model:
\begin{align}
    \dd V_1(t) &= K_1(\theta_1 - V_1(t))\,\dd t + \xi_1\sqrt{V_1(t)}\,\dd W_1(t) \\
    \dd V_2(t) &= K_2(\theta_2 - V_2(t))\,\dd t + \xi_2\sqrt{V_2(t)}\,\dd W_2(t)
\end{align}
The infinitesimal generator involves the first order as well the second order derivative and the covariation. The problem is that in order to have an affine infinitesimal generator it must be
\begin{equation*}
    \expval{Z_1,Z_2} = 0; \qquad \expval{W_1,W_2} = 0; \qquad \expval{Z_1,W_2} = 0; \qquad \expval{Z_2,W_1} = 0;
\end{equation*}
\begin{equation*}
    \expval{Z_1,W_1} = \rho_1; \qquad \expval{Z_2,W_2} = \rho_2.
\end{equation*}
In fact, for example, if $\expval{W_1,W_2}\ne 0$ then there will be a term $\sqrt{V_1V_2}$ and the model will be not affine. However, if we introduce this constraints, there are some consequences. The fact that $W_1 \indep W_2$ implies that the volatility factors are independent. This means that $V_1$ (short term behavior) and $V_2$ (long term behavior) are independent, which is not very reasonable. In particular, if
\begin{equation}
    Y(t) = \ln S(t)
\end{equation}
then its dynamics is
\begin{equation}
    \dd Y(t) = \left(r-\frac{1}{2}(V_1+V_2)\right)\dd t + \sqrt{V_1}\,\dd Z_1 + \sqrt{V_2}\,\dd Z_2.
\end{equation}
Regarding the Fourier pricing, since we know that
\begin{equation*}
    x_1 \indep\, x_2 \qquad\Rightarrow\qquad \mathbb{E}[e^{\lambda_1x_1}]\mathbb{E}[e^{\lambda_2x_2}]
\end{equation*}
then we have
\begin{equation}
    \expect_t[e^{iz\ln S(t)}] = e^{iz\ln S(t) + A(\tau) + B_1(\tau)V_1(t) + B_2(\tau)V_2(t)}
\end{equation}
where $A(\tau)$ is a constant term ($\tau = T-t$) and $B_1$ and $B_2$ are solutions of two independent Riccati ODEs. So, it is possible to find the FT of the asset price in terms of the superpositions of the solutions of the corresponding Riccati equations. The problem is that the (instantaneous) correlation between the noise of the returns and the noise of the volatility of the returns is
\begin{align}
    \notag\text{Corr}(\text{Noise}(\dd Y), \text{Noise}(\text{Vol}(\dd Y))) &= \frac{\dd\expval{Y,V_1+V_2}}{\sqrt{\dd\expval{Y}\dd\expval{V_1+V_2}}} \\
    &=
    \frac{\rho_1\xi_1V_1+\rho_2\xi_2V_2}{\sqrt{V_1+V_2}\sqrt{\xi_1^2V_1 + \xi_2^2V_2}}
    % we can introduce a Brownian motion $\tilde{Z}$ such that we con rewrite \sqrt{V_1}\,\dd Z_1 + \sqrt{V_2}\,\dd Z_2 as $\sqrt{V_1+V_2}\dd \tilde{Z}$
\end{align}
This quantity is also called \emph{leverage} or \emph{stochastic skew}. Recall that in the classic 1-d Heston model the skew (i.e. the slope of the volatility smile) is driven by $\rho$, in fact
\begin{equation*}
    \frac{\dd\expval{Y,V}}{\sqrt{\dd\expval{Y}\dd\expval{V}}} = \frac{\rho\xi V}{\sqrt{V}\sqrt{\xi^2 V}} = \rho
\end{equation*}
In the 2-Heston model this quantity is no more constant, so we can speak about stochastic skew. The presence of stochastic skew is more in line with the real case. The problem is that, in principle, we cannot fit the level ($\sqrt{V_1+V_2}$, i.e. the instantaneous volatility) and the slope of the smile separately, because they are all mixed together. This means that we cannot use ``dedicated" factors.

\lesson{32}{22/05/2020} Now we would like to prove that the Wishart process is affine, in the sense that the LFT is the exponential function of deterministic matrices that satisfy a certain ODE. We will study a specific case in which we consider only the LFT of the Wishart process (without the integral) and we can find the solution without having to solve the Riccati equation. Then we will generalize this result.\\

\subsubsection{Laplace transform of the Wishart process}
Given a $d\times d$ Brownian motion $B(t)$ with i.i.d components, whenever the SDE
\begin{equation}
    \begin{cases}
    \dd S(t) = \sqrt{S(t)}\,\dd B(t) + \dd B(t)^T\sqrt{S(t)} + n\mathds{1}\,\dd t \\
    S(0) = s_0
    \end{cases}
\end{equation}
has a solution in $S_d^+$, the distribution of $S(t)$ for a fixed $t$ is given by its Laplace transform
\begin{equation}\label{lapltrWis}
    \mathbb{E}_{s_0}\left[e^{-\Tr(uS(t))}\right] = \det(\mathds{1}+2tu)^{-\frac{n}{2}}\exp{-\Tr(s_0(\mathds{1}+2tu)^{-1}u)}
\end{equation}
for all $u\in S_d^+$ (in order to have convergence). This is a very strong result, which tells us that the Laplace transform of the Wishart process is given by a deterministic function times the exponential of a linear combination of the elements of $s_0$. This means that the second term is exponentially affine with respect to $S$. So, we can extend to the matrix framework the affinity of the usual CIR process.
\begin{proof}
    The idea behind this proof is to show that the Feynman-Kac formula holds true. That is, if we write the expected value \eqref{lapltrWis} as a function $U(t,s)$, then it satisfies the Feynman-Kac PDE
    \begin{equation*}
    \begin{cases}
        \pdv{U(t,s)}{t} + \mathcal{A}U(t,s) = 0 \\
        U(T,s) = e^{-\Tr(u\cdot s)}
    \end{cases}
    \end{equation*}
    For our convenience, it is better to transform this variable in time to maturity: let's introduce the auxiliary function $V(T-t,s) = U(t,s)$, so that the Feynman-Kac formula becomes
    \begin{equation*}
    \begin{cases}
        -\pdv{V(T-t,s)}{t} + \mathcal{A}V(T-t,s) = 0 \\
        V(0,s) = U(T,s) = e^{-\Tr(u\cdot s)}
    \end{cases}
    \end{equation*}
    Our ``candidate" solution is
    \begin{equation}\label{vts}
        V(t,s) = \det(\mathds{1}+2tu)^{-\frac{n}{2}}\exp{-\Tr(s(\mathds{1}+2tu)^{-1}u)}\tag{$\diamond$}
    \end{equation}
    so if we compute the derivatives they have to satisfy the Feynman-Kac PDE. In order to do that we need to differentiate the determinant of a matrix.
    \begin{lemma}[Jacobi formula]
        The derivative of a deterministic time-dependent matrix $H(t)$ is given by
        \begin{equation*}
            \dv{}{t}\det(H(t)) = \det(H(t))\cdot\Tr\left(H^{-1}(t)\cdot\dv{}{t}H(t)\right)
        \end{equation*}
    \end{lemma}
    \begin{proof}
        The proof of this lemma is based on the fact that if we choose a row $i$ then we can write the determinant as the sum of the elements $H_{ik}$ of this row times the comatrix $\com{H_{ik}}$
        \begin{equation*}
            \det(H) = \sum_k H_{ik}\com{H_{ki}}.
        \end{equation*}
        Recall that the Com($M$) of a $(n\times n)$-matrix $M$ is the $(n \times n)$-matrix whose entry in $(i,k)$ is $(-1)^{i+k}$ times the determinant of the matrix which you get by deleting in $M$ the line $l$ and the row $k$.
    \end{proof}
    In this case
    \begin{equation*}
        \com{H_{ki}} = (-1)^{k+i}\det(H\text{ without row $k$ and column $i$})
    \end{equation*}
    Being the complement of $H_{ki}$, the comatrix $\com{H_{ki}}$ is independent of the elements $H_{ki}$. This means that
    \begin{equation*}
        \pdv{\det(H)}{H_{ij}} = \com{H_{ji}}
    \end{equation*}
    and then
    \begin{align*}
        \dv{}{t}\det(H) &= \sum_j\left(\sum_i \pdv{\det(H)}{H_{ij}}\right)\dv{H_{ij}}{t} \\
        &=
        \sum_j\sum_i\com{H_{ji}}\dv{H_{ij}}{t} \\
        &=
        \Tr(\adj{H}\dd H)
    \end{align*}
    where $\adj{H}$ is the adjugate matrix, i.e. the transpose of the comatrix. \\
    Let's compute the derivatives of \eqref{vts}:
    \begin{align*}
        \pdv{V}{t} &= -\frac{n}{2}\det(\mathds{1}+2tu)^{-\frac{n}{2}-1}\det(\mathds{1}+2tu)\Tr((\mathds{1}+2tu)^{-1}2u)e^{-\Tr(s(\mathds{1}+2tu)^{-1}u)} - \\
        &\qquad -V(t,s)\dv{}{t}\Tr(s(\mathds{1}+2tu)^{-1}u) \\
        &=
        -nV(t,s)\Tr((\mathds{1}+2tu)^{-1}u) + 2V(t,s)\Tr(s(\mathds{1}+2tu)^{-2}u^2)
    \end{align*}
    Now, let's compute the infinitesimal generator:
    \begin{equation*}
        \mathcal{A}V = \Tr(nDV + 2SD^2V)
    \end{equation*}
    where
    \begin{align*}
        DV = \pdv{V}{S_{ij}} &= -V(t,s)D\Tr(S(\mathds{1}+2tu)^{-1}u) \\
        \overset{(a)}&{=}
        -V(t,s)(\mathds{1}+2tu)^{-1}u
    \end{align*}
    where in (a) we used the fact that if we denote $(\mathds{1}+2tu)^{-1}u \equiv B_{ij}$, we have that
    \begin{equation*}
        \Tr(S(\mathds{1}+2tu)^{-1}u) = \sum_{i,j}S_{ij}\cdot B_{ij}
    \end{equation*}
    and the derivative is
    \begin{equation*}
        D\Tr(S(\mathds{1}+2tu)^{-1}u) = \pdv{}{S_{ij}}\sum_{i,j}S_{ij}\cdot B_{ij} = B_{ij} = (\mathds{1}+2tu)^{-1}u
    \end{equation*}
    Then, the second derivative is
    \begin{align*}
        D^2V = V(t,s)(\mathds{1}+2tu)^{-1}u(\mathds{1}+2tu)^{-1}u
    \end{align*}
    So, the infinitesimal generator reads
    \begin{align*}
        \mathcal{A}V &= -nV(t,s)\Tr((\mathds{1}+2tu)^{-1}u) + 2V(t,s)\Tr((\mathds{1}+2tu)^{-1}u(\mathds{1}+2tu)^{-1}u) \\
        \overset{(a)}&{=}
        -nV(t,s)\Tr((\mathds{1}+2tu)^{-1}u) + 2V(t,s)\Tr((\mathds{1}+2tu)^{-1}(\mathds{1}+2tu)^{-1}u)
    \end{align*}
    where in (a) we use the fact that
    \begin{equation*}
        u(\mathds{1}+2tu)^{-1}u = (\mathds{1}+2tu)^{-1}u
    \end{equation*}
    because the two matrices commute. Then, indeed we have
    \begin{equation*}
        -\pdv{V(T-t,s)}{t} + \mathcal{A}V(T-t,s) = 0
    \end{equation*}
    so the Wishart process is affine.
\end{proof} % fine parte 1
Now that we proved that the Wishart process is affine, let's consider some properties of this process in terms of the existence of weak or strong solutions.
\begin{theorem}
    Let $B(t)$ be a $d\times d$ Brownian motion with independent components and let $S(t)$ be the process satisfying the SDE
    \begin{equation}
    \begin{cases}
    \dd S(t) = \sqrt{S(t)}\,\dd B(t) + \dd B(t)^T\sqrt{S(t)} + \alpha\mathds{1}\,\dd t, \qquad \alpha\in\mathbb{R} \\
    S(0) = s_0 \in S_d^+.
    \end{cases}
    \end{equation}
    Then:
    \begin{enumerate}
        \item if $\alpha\ge d-1$ then there exists a weak solution for the SDE (the path-wise uniquess of the solution is not guaranteed);
        \item if $\alpha\ge d+1$ then there exists a strong solution for the SDE, related to the fact that $\det(S(t))>0)$;
        \item if $\alpha\in\{1,\dots,d-2\}$ (Gindikin set) we have no more a Wishart distribution, in fact the distribution il low degenerate and not absolutely continuous with respect to the Lebesgue measure. However, it is still possible to find a weak solution;
        \item if $\alpha\ge d+1$ then the eigenvalues of the matrix are all distinct for all $t$.
        \item if $\alpha\ge d-1$ then $\det(S(t))\ge0$, if $\alpha\ge d+1$ then $\det(S(t))>0$ for all $t$.
        \item if $\alpha\ge d-1$ the eigenvalues of the matrix never collide:
        \begin{equation*}
            \lambda_1(t) > \cdots > \lambda_d(t) \ge 0.
        \end{equation*}
        In this case
        \begin{align*}
            \dd \lambda_i = 2\sqrt{\lambda_i}\,\dd \tilde{B}_i + \left(\alpha+\sum_{k\ne i} \frac{\lambda_i + \lambda_k}{\lambda_i - \lambda_k}\right)\dd t, \qquad \forall i = 1,\dots,d
        \end{align*}
        The term $\tfrac{\lambda_i + \lambda_k}{\lambda_i - \lambda_k}$ is the reason why the eigenvalues never collide (if they collide there is an explosion of the denominator).
    \end{enumerate}
\end{theorem}

\subsubsection{Extension to mean reversion}
Let us consider a $n\times d$-matrix $(X(t))_{n\times d}$ which is solution of the SDE
\begin{equation}
    \begin{cases}\label{xdyn}
        \dd X(t) = \gamma\,\dd B(t) + \beta X(t)\,\dd t \\
        X(0) = x \in M_{n\times d}
    \end{cases}
\end{equation}
where $B(t)$ is a $n\times d$ Brownian motion matrix, $\gamma\in\mathbb{R}$ and $\beta\in\mathbb{R}^-$ (in order to keep the stationarity of the process). This SDE describes a Ornstein-Uhlenbeck process, a stationary Gauss–Markov process which, over time, tends to drift towards its mean function (such a process is called \emph{mean-reverting}). We would like to construct a Wishart process starting from this.
\begin{theorem}
    Set $S(t)=X(t)^TX(t)$ with initial condition $s=x^Tx$. Then
    \begin{equation}
    \begin{cases}
        \dd S(t) = \gamma(\sqrt{S}\,\dd B + \dd B(t)\sqrt{S}) + 2\beta S(t) + n\gamma^2\mathds{1}_d\,\dd t \\
        S(0) = s.
    \end{cases}
    \end{equation}
\end{theorem}
\begin{proof}
    Using Itô and \eqref{xdyn}:
    \begin{align*}
        \dd S(t) &= \dd(X(t)^TX(t)) = (\dd X(t))^TX(t) + X(t)^T\,\dd X(t) + \dd\expval{X^T,X}_t \\
        &=
        \gamma X(t)\,\dd B(t)^T + \beta X(t)^TX(t)\,\dd t + X(t)^T(\gamma\,\dd B(t) + \\
        &\qquad\qquad\qquad\qquad\qquad\qquad\qquad\qquad
        +\beta X(t)\,\dd t) + \gamma^2\,\dd B(t)^T\,\dd B(t) \\
        \intertext{Recall that $\dd B(t)^T\,\dd B(t) = n\mathds{1}_d$ and in order to change the BM multiply and divide for $\sqrt{X(t)^TX(t)}$:}
        &=
        \gamma\left(\sqrt{X(t)^TX(t)}\sqrt{X(t)^TX(t)}^{-1} X(t)^T\,\dd B(t) + \right.\\
        &\qquad
        \left. + \dd B(t)^TX(t)\sqrt{X(t)^TX(t)}\sqrt{X(t)^TX(t)}^{-1}\right) + 2\beta X(t)^TX(t)\,\dd t + \gamma^2 n \mathds{1}_d\,\dd t \\
        \intertext{We can define the Brownian motion $\dd\tilde{B}(t) \equiv \sqrt{X(t)^TX(t)}^{-1} X(t)^T\,\dd B(t)$, in fact $$\Cov(X(t)^T\,\dd B, X(t)^T\,\dd B) = X(t)^T\,\dd B\,\dd B^T X(t) = nX(t)^TX(t)$$ such that $\Cov(\tilde{B},\tilde{B}) = n\mathds{1}$.}
        &=
        \gamma(\sqrt{S(t)}\,\dd\tilde{B}(t) + \dd\tilde{B}^T\sqrt{S(t)}) + 2\beta S(t)\,\dd t + \gamma^2 n\mathds{1}_d\,\dd t
    \end{align*}
\end{proof}
So, we can include a mean reverting drift just by taking a Ornstein-Uhlenbeck process instead of a pure Brownian motion. Another result is the following.
\begin{theorem}
    Let $a\in GL(d)$ (invertible $d\times d$ matrix), $b\in M_d^-$ (eigenvalues with negative real part) and $X(t)$ a Ornstein-Uhlenbeck process defined by
    \begin{equation}
        \begin{cases}
            \dd X(t) = a\,\dd B(t) + X(t)b\,\dd t \\
            X(0) = x
        \end{cases}
    \end{equation}
    where $B(t)$ is a $n\times d$ Brownian motion. If we define the Wishart process
    \begin{equation}
        S(t) = X(t)^TX(t), \qquad s = x^Tx
    \end{equation}
    then
    \begin{equation}
        \dd S(t) = \sqrt{S(t)}\,\dd W(t)\sqrt{a^Ta} + \sqrt{a^Ta}\,\dd W(t)^T\sqrt{S(t)} + (b^TS(t) + S(t)b)\,\dd t + na^Ta\,\dd t
    \end{equation}
    where $na^Ta$ can be replaced by a matrix $C$ such that $C - (d-1)a^Ta \in S^+_d$.
\end{theorem}
\begin{proof}
    \colorbox{cyan}{Homework.}
\end{proof}
Now we have a Wishart process with mean reversion that evolves according to a matrix SDE with matrix coefficients, we can compute the Laplace transform.

\subsubsection{Laplace transform}
Let's denote our Wishart process representing the volatility as $\Sigma(t)$. Its dynamics is given by
\begin{equation}
    \dd\Sigma(t) = (\beta Q^TQ + M\Sigma(t) + \Sigma(t)M^T)\dd t + \sqrt{\Sigma(t)}\,\dd WQ + Q^T(\dd W(t))^T\sqrt{\Sigma(t)}
\end{equation}
where $Q$ is a constant matrix, $M\in M^-$ (to grant stationarity) and $W(t)$ is a $d\times d$ Brownian motion. In order to get at least a weak solution, we have to impose $\beta\ge d-1$. This is the dynamics that guarantee the symmetry and notice that the starting point is a $d\times d$ matrix BM that in principle is not symmetric (all the components are independent). This means that there are $d^2$ sources of noise. However, the information we can deduce from this matrix is much less because the process is symmetric. % boh, non si capisce cosa dice
The infinitesimal generator for this process is
\begin{equation}
    \mathcal{A}_{\Sigma} = \Tr[(\beta Q^TQ + M\Sigma + \Sigma M^T)D + 2\Sigma DQ^TQD]
\end{equation}
\begin{remark}
    If we write the dynamics of the elements of the matrix $\Sigma$ in the case $d=2$, we have
    \begin{align*}
        \dd\Sigma^{11} &= \dots\,\dd t + 2\sigma^{11}(Q_{11}\,\dd W^{11} + Q_{21}\,\dd W^{12}) + 2\sigma^{12}(Q_{11}\,\dd W^{21}+Q_{21}\,\dd W^{22}) \\
        \dd\Sigma^{11} &= \dots\,\dd t + 2\sigma^{12}(Q_{12}\,\dd W^{11} + Q_{22}\,\dd W^{12}) + 2\sigma^{22}(Q_{21}\,\dd W^{21}+Q_{22}\,\dd W^{22})
    \end{align*}
    Then, if we compute the quadratic covariation between the first element in the diagonal and the last one (i.e. the product of the two dynamics), we get
    \begin{align*}
        \dd\expval{\Sigma^{11},\Sigma^{22}}_t &= 4(\sigma^{11}\sigma^{12} + \sigma^{21}\sigma^{22})(Q_{11}Q_{12} + Q_{21}Q_{22})\dd t \\
        &=
        4\Sigma^{12}(Q_{11}Q_{12} + Q_{21}Q_{22})\dd t \ne 0 \text{  if $Q$ is not diagonal.}
    \end{align*}
    When we introduced the first specification of the Wishart process we took $Q = \mathds{1}$, which is diagonal. That's why we were not able to exploit the Wishart property that the covariation between the two positive factors is not trivial (equal to zero). If now $Q$ is not diagonal, the covariation between $\Sigma^{11}>0$ and $\Sigma^{22}>0$ is not zero (like in the usual affine model, where it is the condition to have affinity) and, even more, it is given by the off-diagonal element $\Sigma^{12}$. So, \emph{the Wishart process can be seen as a matrix fluctuating in time in a mean reversion way where there is the joint presence of the volatility factors (diagonal terms) together with the covariance (off-diagonal terms). So, also the correlations will be stochastic, even if we don't loose the analytical tractability.}
\end{remark}

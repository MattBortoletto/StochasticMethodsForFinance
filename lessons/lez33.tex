\subsubsection{Linearization approach for the computation of the Laplace transform}\lesson{33}{27/05/2020}
Now that we know which is the infinitesimal generator for the Wishart process we would like to verify that the process is affine through a different procedure. Instead of solving the Feynman-Kac PDE just by checking if a candidate is a solution we would like to extend the procedure we gave in the context of the CIR model were we doubled the dimension of the problem and used a linear system of ODEs. Basically, we have to adapt that procedure to a matrices framework.\\
If we define the Laplace transform of the Wishart process $\Sigma$ as the conditional expected value
\begin{equation}
    \psi_{\Sigma}(\tau, \Sigma(t)) = \mathbb{E}_t\left[e^{-\Tr(\Lambda\Sigma(T))}\right],
\end{equation}
where $\Lambda\in S^+$ and $\tau = T-t$, then it satisfies the Feynman-Kac formula
\begin{equation}
    \pdv{\psi}{\tau} = \mathcal{A}_{\Sigma}\psi.
\end{equation}
Since we know that the Wishart process is affine, we have a natural candidate solution:
\begin{equation}
    \psi(\tau, \Sigma(t)) = e^{-\Tr(A(\tau)\Sigma(t))+ a(\tau)}.
\end{equation}
where $a(\tau)$ is a scalar deterministic function and $A(\tau)$ is a matrix-valued deterministic function. By replacing, we get
\begin{equation*}
    \begin{cases}
    -\Tr(\dot{A}(\tau)\Sigma(t)) + \dot{a}(\tau) = \Tr[(\beta Q^TQ + M\Sigma + \Sigma M^T)A(\tau) + 2\Sigma A(\tau)Q^TQA(\tau)] \\ % c'è un meno? \Tr[-(\beta Q^TQ + M\Sigma + ...
    A(0) = \Lambda \\
    a(0) = 0
    \end{cases}
\end{equation*}
By identification of the parameters, we have that
\begin{equation*}
    \begin{cases}
        \dot{a}(\tau) = -\Tr(\beta Q^TQ A(\tau)) \\
        a(0) = 0
    \end{cases}
\end{equation*}
and
\begin{equation*}
    \begin{cases}
        \dot{A}(\tau) = A(\tau)M + M^TA(\tau) - 2A(\tau)Q^TQA(\tau) \\
        A(0) = \Lambda
    \end{cases}
\end{equation*}
which is the matrix Riccati ODE.

7:26

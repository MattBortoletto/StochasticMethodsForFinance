\subsubsection{Linearization approach for the computation of the Laplace transform}\lesson{33}{27/05/2020}
Now that we know which is the infinitesimal generator for the Wishart process we would like to verify that the process is affine through a different procedure. Instead of solving the Feynman-Kac PDE just by checking if a candidate is a solution we would like to extend the procedure we gave in the context of the CIR model were we doubled the dimension of the problem and used a linear system of ODEs. Basically, we have to adapt that procedure to a matrices framework.\\
If we define the Laplace transform of the Wishart process $\Sigma$ as the conditional expected value
\begin{equation}
    \psi_{\Sigma}(\tau, \Sigma(t)) = \mathbb{E}_t\left[e^{-\Tr(\Lambda\Sigma(T))}\right],
\end{equation}
where $\Lambda\in S^+$ and $\tau = T-t$, then it satisfies the Feynman-Kac formula
\begin{equation}\label{nmnmnmnnm}
    \pdv{\psi}{\tau} = \mathcal{A}_{\Sigma}\psi.
\end{equation}
Since we know that the Wishart process is affine, we have a natural candidate solution:
\begin{equation}
    \psi(\tau, \Sigma(t)) = e^{-\Tr(A(\tau)\Sigma(t))+ a(\tau)}.
\end{equation}
where $a(\tau)$ is a scalar deterministic function and $A(\tau)$ is a matrix-valued deterministic function. Replacing in \eqref{nmnmnmnnm}, we get
\begin{equation*}
    \begin{cases}
    -\Tr(\dot{A}(\tau)\Sigma(t)) + \dot{a}(\tau) = \Tr[(\beta Q^TQ + M\Sigma + \Sigma M^T)A(\tau) + 2\Sigma A(\tau)Q^TQA(\tau)] \\ % c'è un meno? \Tr[-(\beta Q^TQ + M\Sigma + ...
    A(0) = \Lambda \\
    a(0) = 0.
    \end{cases}
\end{equation*}
By identification of the parameters, we have that
\begin{equation*}
    \begin{cases}
        \dot{a}(\tau) = -\Tr(\beta Q^TQ A(\tau)) \\
        a(0) = 0
    \end{cases}
\end{equation*}
and
\begin{equation*}
    \begin{cases}
        \dot{A}(\tau) = A(\tau)M + M^TA(\tau) - 2A(\tau)Q^TQA(\tau) \\
        A(0) = \Lambda,
    \end{cases}
\end{equation*}
which is the matrix Riccati ODE. Now we linearize by setting
\begin{equation*}
    A(\tau) = F^{-1}(\tau)G(\tau), \qquad F(0) = \mathds{1}, \quad G(0) = \Lambda,
\end{equation*}
where $F(\tau)\in GL_d$ in order to be able to take the inverse. If we take the derivative of $G$ with respect to $\tau$ we get
\begin{equation*}
    \dv{F(\tau)A(\tau)}{\tau} = \dv{F(\tau)}{\tau} A(\tau) + F(\tau)\dv{A(\tau)}{\tau}.
\end{equation*}
Rearranging we get
\begin{equation}
    \dot{G} - \dot{F}A = GM + (FM^T - 2GQ^TQ)A.
\end{equation}
This equation can be seen as the projection of a vector along the direction given by $(\mathds{1}, -A)$. If we identify the coefficients of $A$ and the other terms we get a system of two equations:
\begin{equation}
    \begin{cases}
        \dot{G} = GM \\
        \dot{F} = -FM^T + 2GQ^TQ,
    \end{cases}
\end{equation}
which can be written as
\begin{equation}
    \dv{(G(\tau), F(\tau))}{\tau} = (G(\tau), F(\tau))\underbrace{\mqty(M & 2Q^TQ \\ 0 & -M^T)}_{2d\times 2d}
\end{equation}
This linear system can be solved by taking the exponential of the matrix:
\begin{align}% discorsone sul fatto che questo caso è semplice perchè gli elementi della matrice sono costanti
    \notag(G(\tau), F(\tau)) &= \mqty(G(0) & F(0))\exp{\tau\mqty(M & 2Q^TQ \\ 0 & -M^T)} \\
    &=
    \notag\mqty(\Lambda & \mathds{1})\exp{\tau\mqty(M & 2Q^TQ \\ 0 & -M^T)} \\
    &=
    \notag\mqty(\Lambda & \mathds{1})\mqty(A_{11} & A_{12} \\ A_{21} & A_{22}) \\
    &=
    \mqty(\underbrace{\Lambda A_{11}(\tau) + A_{21}(\tau)}_{G} & \underbrace{\Lambda A_{12}(\tau) + A_{22}(\tau)}_{F}).
\end{align}
So, the solution of the Riccati ODE is
\begin{equation}
    A(\tau) = F^{-1}G = (\Lambda A_{12}(\tau) + A_{22}(\tau))^{-1}(\Lambda A_{11}(\tau) + A_{21}(\tau)).
\end{equation}
Then, we can find $a(\tau)$:
\begin{align*}
    \dot{a}(\tau) &= -\beta\Tr(Q^TQA(\tau)) \\
    &=
    -\beta\Tr(Q^TQF^{-1}G) \\
    \intertext{From $\dot{F} = -FM^T + 2GQ^TQ$ we have that $G = \tfrac{1}{2}(\dot{F}+FM^T)(Q^TQ)^{-1}$:}
    &=
    -\frac{\beta}{2}\Tr(Q^TQF^{-1}(F+FM^T)(Q^TQ)^{-1}) \\
    \intertext{Recalling that $\partial_z\Tr(\ln A(z)) = \Tr((\partial_z A)A^{-1})$ we recognize the logarithm of $F$:}
    &=
    -\frac{\beta}{2}\Tr(\dd\ln(F)+M^T)
\end{align*}
By integrating we get
\begin{equation}
    a(\tau) = -\frac{\beta}{2}\Tr(\ln F(\tau) + M^T\tau).
\end{equation}
In conclusion, we have the ``explicit" (numerical) Laplace transform. Now we have to extend the Laplace transform of the Wishart to the case in which there is also a time integral of the process.

\subsubsection{Extension to the case of time integral}
Consider the case in which the Laplace transform includes also a time integral:
\begin{equation}
    \psi_{\Sigma}(\tau, \Sigma(t)) = \mathbb{E}_t\left[e^{-\Tr(\Lambda\Sigma(T))+\Gamma\int_t^T \Sigma(s)\,\dd s}\right],
\end{equation}
where $\Lambda, \Gamma \in S^+$. According to the Feynman-Kac methodology $\psi$ satisfies the PDE
\begin{equation}
    \pdv{\psi}{\tau} = \mathcal{A}_{\Sigma}\psi - \Tr(\Gamma\Sigma)\psi.
\end{equation}
A candidate solution is given by
\begin{equation}
    \psi(\tau,\Sigma(t)) = e^{-\Tr(A(\tau)\Sigma(t)) + a(\tau)}.
\end{equation}
These equations lead to the following Riccati matrix ODEs:
\begin{align*}
    \begin{cases}
        \dot{A}(\tau) = A(\tau)M + M^TA(\tau) - 2A(\tau)Q^TQA(\tau) + \Gamma \\
        A(0) = \Lambda
    \end{cases}
\end{align*}
\begin{align*}
    \begin{cases}
        \dot{a}(\tau) = -\beta\Tr(Q^TQA(\tau)) \\
        a(0) = 0.
    \end{cases}
\end{align*}
As before, we set
\begin{equation*}
    A(\tau) = F^{-1}(\tau)G(\tau)
\end{equation*}
and we differentiate:
\begin{align*}
    \dv{F\cdot A}{\tau} = \dot{F}A + F\dot{A} = \dot{F}A + F(AM+M^TA - 2AQ^TQA + \Gamma).
\end{align*}
So, we can write
\begin{align*}
    \dot{G} - \dot{F}A = GM + (FM^T - 2GQ^TQ)A + F\Gamma,
\end{align*}
which leads to the linear system
\begin{align*}
    \begin{cases}
        \dot{G} = GM + F\Gamma \\
        \dot{F} = -FM^T + 2GQ^TQ,
    \end{cases}
\end{align*}
which can be written as
\begin{equation*}
    \mqty(\dot{G} & \dot{F}) = \mqty(G & F)\mqty(M & 2Q^TQ \\ \Gamma & -M^T).
\end{equation*}
The solution is
\begin{align}
    A(\tau) = (\Lambda A_{12} + A_{22})^{-1}(\Lambda A_{11} + A_{21}),
\end{align}
where
\begin{equation}
    \mqty(A_{11} & A_{12} \\ A_{21} & A_{22}) = \exp{\tau\mqty(M & 2Q^TQ \\ \Gamma & -M^T)}.
\end{equation} % fine parte 1
Then we can find $a(\tau)$ as before.

\subsection{A multi-factor Wishart volatility model}
Now that we have the Laplace transform of the Wishart process we can consider a model in which the Wishart process plays the role of the volatility. We will see that this model extends the Heston model in a non-trivial way.\\
As usual, the starting point is the dynamics of the underlying under the risk neutral measure:
\begin{equation}
    \frac{\dd S(t)}{S(t)} = r\,\dd t + \Tr(\sqrt{\Sigma(t)} \,\dd Z(t)),
\end{equation}
where $Z(t)$ is a Brownian motion that drives the asset returns, which is partially correlated with $W(t)$, and
\begin{equation}
    \dd\Sigma(t) = (\beta Q^TQ + M\Sigma(t) + \Sigma(t)M^T)\dd t + \sqrt{\Sigma(t)}\,\dd W(t)Q + Q^T\,\dd W^T(t)\sqrt{\Sigma(t)},
\end{equation}
where $Q^TQ$ is invertible, $M$ is negative semi-definite (in order to have stationarity), $\beta\ge n-1$ (in order to have at least a weak solution) and $\Sigma(t)\in S^+_n$.\\
Both $W$ and $Z$ are $(n\times n)$-Brownian motions, so there are $n^2$ possibilities of correlation. However, we have to keep the affinity property, for example setting
\begin{equation}
    Z(t) = W(t)\cdot R^T + B(t)\sqrt{\mathds{1}-RR^T},
\end{equation}
where $R\in M_{n\times n}$ and $B(t)$ is a Brownian motion such that $W(t)\indep\, B(t)$.
\begin{proposition}
    If we define $Z$ as above, then it is a (matrix) Brownian motion.
\end{proposition}
\begin{proof}
    We have to show that $\forall \alpha_1,\alpha_2\in\mathbb{R}^n$
    \begin{align*}
        \Cov_t(\dd Z\alpha_1, \dd Z\alpha_2) = \mathbb{E}_t[(\dd Z\alpha_1)(\dd Z\alpha_2)^T] = \alpha_1^T\alpha_2\mathds{1}_n\,\dd t
    \end{align*}
    By replacing the definition of $Z$ we get
    \begin{align*}
        \Cov_t(\dd Z\alpha_1, \dd Z\alpha_2) &= \mathbb{E}_t[(\dd WM^T\alpha_1 + \dd B\sqrt{\mathds{1}-RR^T}\alpha_1)(\dd WM^T\alpha_2 + \dd B\sqrt{\mathds{1}-RR^T}\alpha_2)^T] \\
        \intertext{Use the fact that $W(t)\indep\, B(t)$:}
        &=
        \Cov_t(\dd WR^T\alpha_1, \dd WR^T\alpha_2) + \Cov_t(\dd B\sqrt{\mathds{1}-RR^T}\alpha_1, \dd B\sqrt{\mathds{1}-RR^T}\alpha_2) \\
        \intertext{Now use the fact that $W$ and $B$ are true (matrix) Brownian motions:}
        &=
        \alpha_1^T RR^T \alpha_2\mathds{1}\,\dd t + \alpha_1^T(\mathds{1}-RR^T)\alpha_2\mathds{1}\,\dd t \\
        &=
        \alpha_1^T\alpha_2\mathds{1}\,\dd t
    \end{align*}
    So $Z$ is a true Brownian motion.
\end{proof}
\begin{remark}
    This is not the only possibility to correlate $Z$ with $W$. This parametrization is quite natural because it guarantees linearity and it is analogous with what we did in the scalar case.
\end{remark}
If we are interested in pricing a call:
\begin{align*}
    price_t(\call) = -\frac{e^{-r(T-t)}}{2\pi}\int_{\mathcal{Z}}\frac{K^{iz+1}}{z^2+iz}\hat{f}(-z)\,\dd z
\end{align*}
we need
\begin{equation*}
    \hat{f}(z) = \mathbb{E}_t[e^{iz\ln S(T)}].
\end{equation*}
Let's introduce the function
\begin{align}
    \psi_{\gamma, t}(\tau) = \mathbb{E}_t\left[e^{\gamma\ln S(t+\tau)}\right] = \mathbb{E}_t\left[e^{\gamma Y(t+\tau)}\right],
\end{align}
where $Y(t) = \ln S(t)$ plays the role of the log-price. From the dynamics of the asset price we can deduce the dynamics of $Y(t)$:
\begin{equation*}
    \dd Y(t) = \left(r-\frac{1}{2}\Tr(\Sigma(t))\right)\dd t + \Tr(\sqrt{\Sigma(t)}(\dd W(t)R^T+\dd B(t)\sqrt{\mathds{1}-RR^T})).
\end{equation*}
By Feynman-Kac we have that $\psi$ satisfies the PDE
\begin{equation}
    \begin{cases}
        \pdv{\psi_{\gamma,t}}{\tau} = \mathcal{A}_{Y,\Sigma}\psi_{\gamma,t} \\
        \psi_{\gamma,t}(0) = e^{\gamma Y(t)},
    \end{cases}
\end{equation}
where $\mathcal{A}_{Y,\Sigma}$ is the joint infinitesimal generator of the whole stochastic process.
\begin{proposition}
    The multi-factor Wishart model is affine and its infinitesimal generator is given by
    \begin{align}
        \notag\mathcal{A}_{Y,\Sigma} &= \underbrace{\left(r-\frac{1}{2}\Tr(\Sigma)\right)\pdv{}{y} + \frac{1}{2}\Tr(\Sigma)\pdv[2]{}{y}}_{\mathcal{A}_Y} + \underbrace{\Tr((\beta Q^TQ + 2M\Sigma)D + 2\Sigma DQ^TQD)}_{\mathcal{A}_{\Sigma}} + \\
        % 2M\Sigma e non M\Sigma + \SigmaM^T perchè sfruttiamo la commutatività
        &\qquad +
        \underbrace{2\Tr(\Sigma RQD)\pdv{}{y}}_{\mathcal{A}_{\expval{Y,\Sigma}}}. % Y is a scalar (it is the log-price) and \Sigma is a matrix. This is the reason why here we have a matrix operator D and a scalar operator d/dy.
    \end{align}
\end{proposition}
\begin{proof}
    We have already proved the expressions for the infinitesimal generator for $Y$ and $\Sigma$, so we only need to study the coviariation term. Recalling that $(\sqrt{\Sigma})_{ij} = \sigma_{ij}$
    \begin{align*}
        \dd\expval{\Sigma^{ij},Y}_t &= \mathbb{E}_t\left[\sum_{l,k =1}^n\sigma_t^{il}\dd W_{lk}Q_{kj} + \sum_{l,k =1}^n\sigma_t^{jl}\dd W_{lk}Q_{ki}\right]\left(\sum_{i,k,h=1}^n \sigma_t^{lk}\dd W_{kh}Q_{lh}\right) \\
        &=
        \sum_{l,k,h=1}^n (\sigma^{il}Q_{kj}+\sigma^{jl}Q_{ki})\sigma^{hl}R_{hk}\,\dd t \\
        &=
        \sum_{k,h=1}^n\left[\hlc{mypink}{\left(\sum_{l=1}^n \sigma^{il}\sigma^{kl}\right)}Q_{kj} + \hlc{mylightblue}{\left(\sum_{l=1}^n \sigma^{jl}\sigma^{hl}\right)}Q_{hi}\right]R_{hk}\,\dd t \\
        &=
        \sum_{k,h=1}^n\left[\hlc{mypink}{\Sigma^{ih}}Q_{kj} + \hlc{mylightblue}{\Sigma^{jk}}Q_{hi}\right]R_{hk}\,\dd t \\
        &=
        \text{coefficient of } \pdv{}{X_{ij}}{y}
    \end{align*}
    In fact, note that if in
    \begin{align*}
        2\Tr(\Sigma RQD)\pdv{}{y} = 2\sum_{i,j,k,h=1}^n D^{ij}\Sigma^{jh}R_{hk}Q_{ki}\pdv{}{y}
    \end{align*}
    we fix $i$ and $j$ we find the expression above.
\end{proof}

\subsubsection{Linearization approach for the computation of the Laplace transform}\lesson{33}{27/05/2020}
Now that we know which is the infinitesimal generator for the Wishart process we would like to verify that the process is affine through a different procedure. Instead of solving the Feynman-Kac PDE just by checking if a candidate is a solution we would like to extend the procedure we gave in the context of the CIR model were we doubled the dimension of the problem and used a linear system of ODEs. Basically, we have to adapt that procedure to a matrices framework.\\
If we define the Laplace transform of the Wishart process $\Sigma$ as the conditional expected value
\begin{equation}
    \psi_{\Sigma}(\tau, \Sigma(t)) = \mathbb{E}_t\left[e^{-\Tr(\Lambda\Sigma(T))}\right],
\end{equation}
where $\Lambda\in S^+$ and $\tau = T-t$, then it satisfies the Feynman-Kac formula
\begin{equation}
    \pdv{\psi}{\tau} = \mathcal{A}_{\Sigma}\psi.
\end{equation}
Since we know that the Wishart process is affine, we have a natural candidate solution:
\begin{equation}
    \psi(\tau, \Sigma(t)) = e^{-\Tr(A(\tau)\Sigma(t))+ a(\tau)}.
\end{equation}
where $a(\tau)$ is a scalar deterministic function and $A(\tau)$ is a matrix-valued deterministic function. By replacing, we get
\begin{equation*}
    \begin{cases}
    -\Tr(\dot{A}(\tau)\Sigma(t)) + \dot{a}(\tau) = \Tr[(\beta Q^TQ + M\Sigma + \Sigma M^T)A(\tau) + 2\Sigma A(\tau)Q^TQA(\tau)] \\ % c'è un meno? \Tr[-(\beta Q^TQ + M\Sigma + ...
    A(0) = \Lambda \\
    a(0) = 0.
    \end{cases}
\end{equation*}
By identification of the parameters, we have that
\begin{equation*}
    \begin{cases}
        \dot{a}(\tau) = -\Tr(\beta Q^TQ A(\tau)) \\
        a(0) = 0
    \end{cases}
\end{equation*}
and
\begin{equation*}
    \begin{cases}
        \dot{A}(\tau) = A(\tau)M + M^TA(\tau) - 2A(\tau)Q^TQA(\tau) \\
        A(0) = \Lambda,
    \end{cases}
\end{equation*}
which is the matrix Riccati ODE. Now we linearize by setting
\begin{equation*}
    A(\tau) = F^{-1}(\tau)G(\tau), \qquad F(0) = \mathds{1}, \quad G(0) = \Lambda,
\end{equation*}
where $F(\tau)\in GL_d$ in order to be able to take the inverse. If we take the derivative of $G$ with respect to $\tau$ we get
\begin{equation*}
    \dv{F(\tau)A(\tau)}{\tau} = \dv{F(\tau)}{\tau} A(\tau) + F(\tau)\dv{A(\tau)}{\tau}.
\end{equation*}
Rearranging we get
\begin{equation}
    \dot{G} - \dot{F}A = GM + (FM^T - 2GQ^TQ)A.
\end{equation}
This equation can be seen as the projection of a vector along the direction given by $(\mathds{1}, -A)$. If we identify the coefficients of $A$ and the other terms we get a system of two equations:
\begin{equation}
    \begin{cases}
        \dot{G} = GM \\
        \dot{F} = -FM^T + 2GQ^TQ,
    \end{cases}
\end{equation}
which can be written as
\begin{equation}
    \dv{}{\tau} (G(\tau), F(\tau)) = (G(\tau), F(\tau))\underbrace{\mqty(M & 2Q^tQ \\ 0 & -M^T)}_{2d\times 2d}
\end{equation}
This linear system cn be solved by taking the exponential of the matrix:
\begin{align}% discosone sul fatto che questo caso è semplice perchè gli elementi della matrice sono costanti
    \notag(G(\tau), F(\tau)) &= (G(0), F(0))\exp{\tau\mqty(M & 2Q^tQ \\ 0 & -M^T)} \\
    &=
    \notag(\Lambda, \mathds{1})\exp{\tau\mqty(M & 2Q^tQ \\ 0 & -M^T)} \\
    &=
    \notag(\Lambda, \mathds{1})\mqty(A_{11} & A_{12} \\ A_{21} & A_{22}) \\
    &=
    (\underbrace{\Lambda A_{11}(\tau) + A_{21}(\tau)}_{G}, \underbrace{\Lambda A_{12}(\tau) + A_{22}(\tau)}_{F}).
\end{align}
So, the solution of the Riccati ODE is
\begin{equation}
    A(\tau) = F^{-1}G = (\Lambda A_{12}(\tau) + A_{22}(\tau))^{-1}(\Lambda A_{11}(\tau) + A_{21}(\tau)).
\end{equation}
Then, we can find $a(\tau)$:
\begin{align*}
    \dot{a}(\tau) &= -\beta\Tr(Q^TQA(\tau)) \\
    &=
    -\beta\Tr(Q^TQF^{-1}G) \\
    &=
    \intertext{From $\dot{F} = -FM^T + 2GQ^TQ$ we have that $G = \tfrac{1}{2}(\dot{F}+FM^T)(Q^TQ)^{-1}$}
    -\frac{\beta}{2}\Tr(Q^TQF^{-1}(F+FM^T)(Q^TQ)^{-1}) \\
    \intertext{Recalling that $\partial_z\Tr(\ln A(z)) = \Tr((\partial_z A)A^{-1})$ we recognize the logarithm of $F$:}
    &=
    -\frac{\beta}{2}\Tr(\dd\ln(F)+M^T)
\end{align*}
By integrating we get
\begin{equation}
    a(\tau) = -\frac{\beta}{2}\Tr(\ln F(\tau) + M^T\tau).
\end{equation}
In conclusion, we have the ``explicit" (numerical) Laplace transform.

21:20
